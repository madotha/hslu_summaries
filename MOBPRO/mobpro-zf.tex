\documentclass[a4paper]{article}

\usepackage[ngerman]{babel}
\usepackage[utf8]{inputenc}
\usepackage[T1]{fontenc}

\usepackage[scaled]{helvet}
\renewcommand{\familydefault}{\sfdefault}

\usepackage[margin=80pt]{geometry}

\usepackage{multicol}
\usepackage{graphicx}
\usepackage{caption}

\usepackage[table,xcdraw]{xcolor}
\usepackage{courier}
\usepackage{subcaption}

\usepackage{enumitem}
\setlist{nosep}

\usepackage [autostyle]{csquotes}
\MakeOuterQuote{"}

\usepackage[hidelinks]{hyperref}
\hypersetup{colorlinks=false}

\usepackage{listings}
\usepackage{color}
\definecolor{dkgreen}{rgb}{0,0.6,0}
\definecolor{gray}{rgb}{0.5,0.5,0.5}
\definecolor{mauve}{rgb}{0.58,0,0.82}

\lstset{frame=tb,
	language=Java,
	aboveskip=3mm,
	belowskip=3mm,
	showstringspaces=false,
	columns=flexible,
	basicstyle={\small\ttfamily},
	numbers=left,
	numberstyle=\tiny\color{gray},
	keywordstyle=\color{blue},
	commentstyle=\color{dkgreen},
	stringstyle=\color{mauve},
	breaklines=true,
	breakatwhitespace=true,
	tabsize=3
}


\title{\textbf{MOBPRO - Mobile Programming\\
Zusammenfassung FS 2019}}
\date{\today}
\author{Maurin D. Thalmann}

\begin{document}
	\pagenumbering{gobble}
	\maketitle
	\newpage
	\pagenumbering{arabic}
	\tableofcontents
	\newpage	
	
\section{Android 1 - Grundlagen}
	Informationen zur Androidprogrammierung können stets dem Android Developer Guide entnommen werden unter: \textit{\href{https://developer.android.com/}{developer.android.com}}
	Apps sollen grundsätzlich gegen das aktuellste API entwickelt werden, aktuell API Level 28 Android 9 "Pie".
	Im Gradle-Build-Skript werden deshalb folgende SDK-Versionen festgehalten:
\vspace{1em}
	\begin{description}
		\item[\textit{minSdkVersion}] Mindestanforderung an die SDK, Minimum-Version
		\item[\textit{targetSdkVersion}] Ziel-SDK-Version, auf welcher die App lauffähig sein soll
		\item[\textit{compileSdkVersion}] Version mit welcher die App (APK) erstellt wird, meist gleich der Target-Version	
	\end{description}
\vspace{1em}
	\textbf{ART (Android Runtime)} verwaltet Applikationen bzw. deren einzelne Komponenten:
	\begin{itemize}
		\item Komponente kann andere Komponente mit Intent-Mechanismus aufrufen
		\item Komponenten müssen beim System registriert werden (teilweise mit Rechten = Privileges)
		\item System verwaltet Lebenszyklus von Komponenten: Gestartet, Pausiert, Aktiv, Gestoppt, etc.
	\end{itemize}

	\subsection{Komponenten}
		Applikationen sind aus Komponenten aufgebaut, die App verwendet dabei eigene Komponenten (min. eine) oder Komponenten von anderen, existierenden Applikationen.
	\begin{table}[h!]
		\begin{tabular}{ l | p{11cm} }
			\textbf{\textit{Name}}               & \textbf{\textit{Beschreibung}} \\
			\hline
			\textbf{Activity}           & UI-Komponente, entspricht typischerweise einem Bildschirm \\
			\textbf{Service}            & Komponente ohne UI, Dienst läuft typischerweise im Hintergrund \\
			\textbf{Broadcast Receiver} & Event-Handler, welche auf App-interne oder systemweite Broadcast-Nachrichten reagieren \\
			\textbf{Content Provider}   & Komponente, welche Datenaustausch zwischen versch. Applikationen ermöglicht
		\end{tabular}
	\end{table}
	
	\noindent
	\textbf{Activity} entspricht einem Bildschirm, stellt UI-Widgets dar, reagiert auf Benutzer-Eingabe \& -Ereignisse. Eine App besteht meist aus mehreren Activities / Bildschirmen, die auf einem "Stack" liegen. \\
	Basisklasse: \textit{android.app.Activity} \\
	\textbf{Service} läuft typischerweise im Hintergrund für unbeschränkte Zeit, hat keine graphische Benutzer\-schnittstelle (UI), ein UI für ein Service wird immer von einer Activity dargestellt. \\
	Basisklasse: \textit{android.app.Service} \\
	\textbf{Broadcast Receiver} ist eine Komponente, welche Broadcast-Nachrichten empfängt und darauf reagiert. Viele Broadcasts stammen vom System (Neue Zeitzone, Akku fast leer,...), App kann aber auch interne Broadcasts versenden. \\
	Basisklasse: \textit{android.content.BroadcastReceiver}\\
	\textbf{Content Provider} ist die einzige \textit{direkte} Möglichkeit zum Datenaustausch zwischen Android-Apps. Bieten Standard-API für Suchen, Löschen, Aktualisieren und Einfügen von Daten. \\
	Basisklasse: \textit{android.content.ContentProvider}
	
	\newpage
	\subsection{Das Android-Manifest}
	\textbf{AndroidManifest.xml} dient dazu, alle Komponenten einer Applikation dem System bekannt zu geben. Es enthält Informationen über Komponenten der Applikation, statische Rechte (Privileges), Liste mit Erlaubnissen (Permissions), ggf. Einschränkungen für Aufrufe (Intent-Filter). Es beschreibt die statischen Eigenschaften einer Applikation, beispielsweise: \\
	\textit{(Diese Infos werden bei der App-Installation im System registriert, zusätzliche Infos (Version, ID, etc.) befinden sich im Gradle-Build-Skript (können build-abhängig sein))}
	
	\begin{itemize}
		\item Java-Package-Name
		\item Benötigte Rechte (Internet, Kontakte, usw.)
		\item Deklaration der Komponenten
		\begin{itemize}
			\item Activities, Services, Broadcast Receivers, Content Providers
			\item Name (+ Basis-Package = Java Klasse)
			\item Anforderungen für Aufruf (Intent) für A, S, BR
			\item Format der gelieferten Daten für CP
		\end{itemize}
	\end{itemize}
	\begin{figure}[htb!]
		\centering
		\includegraphics[width=12cm]{img/manifestxml.jpg}
		\caption{Beispiel eines Android-Manifests}
		\label{fig:manifestxml}
	\end{figure}

	\subsection{Activities \& Aufruf mit Intents}
	Zwischen Komponenten herrscht das Prinzip der losen Kopplung:
	\begin{itemize}
		\item Komponenten rufen andere Komponenten über Intents (= Nachrichten) auf
		\item Offene Kommunikation: Sender weiss nicht ob Empfänger existiert
		\item Parameterübergabe als Strings (untypisiert)
		\item Parameter: von Empfänger geprüft, geparst \& interpretiert (oder ignoriert)
		\item Keine expliziten Abhängigkeiten $\rightarrow$ Robuste Systemarchitektur
	\end{itemize}
	\newpage
	\begin{figure}[htb!]
		\centering
		\includegraphics[width=7.5cm]{img/intents_comm.png}
		\caption{Kommunikation zwischen Komponenten mit Intents}
		\label{fig:intents_comm}	
	\end{figure}
	\noindent
	Intents werden benutzt, um Komponenten zu benachrichtigen oder um Kontrolle zu übergeben. Es gibt folgende zwei Arten von Intents:
	\begin{description}
		\item[Explizite Intents] adressieren eine Komponente direkt
		\item[Implizite Intents] beschreiben einen geeigneten Empfänger
	\end{description}
	\textbf{WICHTIG:} Activities müssen immer im Manifest deklariert werden, da sie sonst nicht als "public" gelten und eine Exception schmeissen. Das geht auch ganz einfach folgendermassen im Manifest unter "application":
	\begin{lstlisting}
	<activity android:name=".Sender" />
	<activity android:name=".Receiver" />
	\end{lstlisting}
	
	\subsubsection{Beispielaufruf Expliziter Intent}
	\textbf{Sender Activity:}
	\begin{lstlisting}
	public void onClickSendBtn(final View btn) {
		Intent intent = new Intent(this, Receiver.class); 
		// Receiver.class ist hier der explizite Empfaenger
		intent.putExtra("msg", "Hello World!");
		startActivity(intent);
	}
	\end{lstlisting}
	\textbf{Receiver Activity:}
	\begin{lstlisting}
	public void onCreate(Bundle savedInstanceState) {
		// ...
		Intent intent = getIntent();
		String msg = intent.getExtras().getString("msg");
		displayMessage(msg);
	}
	\end{lstlisting}
	
	\newpage
	
	\subsubsection{Beispielaufruf Impliziter Intent}
	\textbf{Sender Activity:}
	\begin{lstlisting}
	Intent browserCall = new Intent();
	browserCall.setAction(Intent.ACTION_VIEW);
	browserCall.setData(Uri.parse("http://www.hslu.ch"));
	startActivity(browserCall);	
	\end{lstlisting}
	\textit{ACTION\_VIEW} ist hierbei kein expliziter Empfängertyp, sondern nur eine gewünschte Aktion. Die mitgege\-bene URL wird auch ein \textit{Call Parameter} genannt. Gesucht ist in diesem Fall eine Komponente, welche eine URL anzeigen/verwenden kann.\\
	\subsection{Activities \& Subactivities}
	\textbf{Activity Back Stack:} Activities liegen aufeinander wie ein Stapel Karten, neuste Activity zuoberst und in der Regel ist nur diese sichtbar (Durch Transparenz sind hier Ausnahmen möglich).
	Durch "back" oder "finish" wird die oberste Karte entfernt und man kehrt zur zweitletzten Activity zurück. Mehrere Instanzen derselben Activity wären mehrere solche Karten, das Verhalten kann jedoch konfiguriert werden (z.Bsp. maximal eine Instant, mehrere Activities öffnen, etc.)\\
	\textbf{(Sub-)Activities und Rückgabewerte:} Eine Activity kann Rückgabewerte einer anderen (Sub-)Activity erhalten.
	\begin{lstlisting}
	// 1. Aufruf der SubActivity mit
	startActivityForResult(intent, requestId)
	
	// 2. SubActivity setzt am Ende Resultat mit
	setResult(resultCode, intent) // intent als Wrapper fuer Rueckgabewerte
	
	// 3. SubActiity beendet sich mit
	finish()
	
	// 4. Nach Beendung der SubActivity wird folgendes im Aufrufer aufgerufen:
	onActivityResult(requestId, resultCode, intent)
	// resultCode: RESULT_OK, RESULT_CANCELLED
	\end{lstlisting}
	\subsection{Lebenszyklus \& Zustände von Applikationen/Activities}
	Das System kann Applikationen bei knappem Speicher ohne Vorwarnung terminieren (nur Activities im Hintergrund, dies geschieht unbemerkt vom User, die App wird bei Zurücknavigation wiederhergestellt). Eine Applikation kann ihren Lebenszyklus demnach nicht kontrollieren und muss in der Lage sein, ihren Zustand speichern und wieder laden zu können. Applikationen durchlaufen mehrere Zustände in ihrem Lebenszyklus, Zustandsübergänge rufen Callback-Methoden auf (welche von uns überschrieben werden können.\\
	
	\noindent
	\textbf{Activity-Zustände:}
	\begin{table} [h!]
		\begin{tabular}{ c | p{10cm} }
			\textbf{Zustand} & \textbf{Beschreibung} \\ \hline
			\textbf{Running} & Die Activity ist im Vordergrund auf dem Bildschirm (zuoberst auf dem Activity-Stack für die aktuelle Aufgabe). \\ \hline
			\textbf{Paused} & Die Activity hat den Fokus verloren, ist aber immer noch sichtbar für den Benutzer. \\ \hline
			\textbf{Stopped} & Die Activity ist komplett verdeckt von einer andern Activity. Der Zustand der Activity bleibt jedoch erhalten.
		\end{tabular}
	\end{table}
	\newpage
	\subsubsection{Lifecycle einer Applikation}
	\begin{figure}[h!]
		\centering
		\includegraphics[width=13cm]{img/lifecycle.jpg}
		\caption{Lifecycle einer Applikation}
		\label{fig:lifecycle}
	\end{figure}
	\begin{figure}[h!]
		\centering
		\includegraphics[width=11cm]{img/lifetime.jpg}
		\caption{Lebenszeiten der einzelnen App-Zustände}
		\label{fig:lifetime}
	\end{figure}
	\newpage
	\subsection{Charakterisierung einer Activity}
	\begin{itemize}
		\item Muss im Manifest deklariert werden
		\item GUI-Controller
		\begin{itemize}
			\item Repräsentiert eine Applikations-/Bildschirmseite
			\item Definiert Seitenlayout und GUI-Komponenten
			\item Kann aus Fragmenten ( = "Sub-Activities") aufgebaut sein
			\item Reagiert auf Benutzereingaben
			\item Beinhaltet Applikationslogik für dargestellte Seite
		\end{itemize}
	\end{itemize}
	\textbf{Beispiel einer Activity:}
	\begin{lstlisting}
	public class Demo extends Activity {
		// Called when the Activity is first created
		public void onCreate(Bundle savedInstanceState) {
			super.onCreate(savedInstanceState);
			setContentView(R.layout.main); // Definiert Layout und UI
		} 
	}	
	\end{lstlisting}
	\subsubsection{Zustandsänderung - Hook-Methoden}
	Das System benachrichtigt Activities durch Aufruf einer der folgenden Methoden der Klasse \textit{Activity}:
	\begin{itemize}
		\item void onCreate(Bundle savedInstanceState)
		\item void onStart() / void onRestart()
		\item void onResume()
		\item void onPause() $\rightarrow$ \textit{bspw. Animation stoppen}
		\item void onStop()
		\item void onDestroy() $\rightarrow$ \textit{bspw. Ressourcen freigeben}
	\end{itemize}
	Durch das Überschreiben dieser Methoden können wir uns in den Lebenszyklus einklinken. Immer \textbf{super()} aufrufen, sonst wirft es eine Exception.
	\newpage
	\subsection{Android - Hinter den Kulissen}
	\begin{figure}[h!]
		\centering
		\includegraphics[width=11cm]{img/androidstack.jpg}
		\caption{Der Android-Stack}
		\label{fig:androstack}
	\end{figure}
	\begin{itemize}
		\item \textbf{Linux-Kernel:} OS, FS, Security, Drivers, ...
		\item \textbf{HAL (Hardware Abstraction Layer):} Camera-, Sensor-, ... Abstraktion
		\item \textbf{ART} (Android Runtime)
		\begin{itemize}
			\item Jede App in eigenem Prozess
			\item Optimiert für mehrere JVM auf low-memory Geräten
			\item Eigenes Bytecode-Format (Crosscompiling)
			\item JIT und AOT Support
		\end{itemize}
		\item \textbf{Native C/C++ Libriaries:} Zugriff via Android NDK
		\item \textbf{Android Framework:} Android Java API
		\item \textbf{Applications:} System- und eigene Apps
	\end{itemize}
	\newpage
	\subsubsection{Android-Security-Konzept}
	\textbf{Sandbox-Konzept:} 
	\begin{itemize}
		\item Jede laufende Android-Anwendung hat seinen eigenen Prozess, Benutzer, ART-Instanz, Heap und Dateisystembereich $\rightarrow$ jedes App hat eigenen Linux-User
		\item Das Berechtigungssystem von Linux ist Benutzer-basiert, es betrifft deshalb sowohl den Speicherzugriff wie auch das Dateisystem. 
		\item Anwendungen signieren: erschwert Code-Manipulationen und erlaubt das Teilen einer Sandbox bei gleicher sharedUser-ID
		\item Berechtigungen werden im Manifest deklariert, kontrollierte Öffnung der Sandbox-Restriktionen
	\end{itemize}
	\begin{figure}[htb!]
		\centering
		\includegraphics[width=12cm]{img/securitymodel.jpg}
		\caption{Android Security-Modell}
		\label{fig:secumodel}
	\end{figure}
	
	\newpage
\section{Android 2 - Benutzerschnittstellen}

\subsection{GUI einer Activity}

GUI wird als XML definiert, der Name resultiert in einer Konstante: \textit{\textbf{R}.layout.xxx}. Diese wird im \textit{onCreate()} einer Activity mit \textit{setContentView()} angegeben.

\begin{figure}[htb!]
	\centering
	\includegraphics[width=10cm]{img/gui_xml_example.jpg}
	\caption{Beispiel eines XML für ein Layout}
	\label{fig:guixmlexamp}
\end{figure}
\noindent
Je nach Layout müssen die Elemente unterschiedlich konfiguriert werden, was bei der Arbeit mit dem Layout-Editor nicht offensichtlich, aber trotzdem gut zu wissen ist. \\
Ein Android-UI ist hierarchisch aufgebaut und besteht aus \textbf{ViewGroups} (Cointainer für Views oder weitere ViewGroups, angeordnet durch Layout) und \textbf{Views} (Widgets). Sollte auf unterschiedlichen Bildschirmgrössen gleich aussehen (Elemente deshalb \textbf{relativ} und nicht absolut positionieren)

\begin{figure}[htb!]
	\centering
	\includegraphics[width=10cm]{img/xml_layouts.jpg}
	\caption{Layout-Varianten bei Android}
	\label{fig:xmllayouts}
\end{figure}
\noindent
Schachtelung möglich, aber nicht effizient, wenn möglich immer das Constraint-Layout verwenden. Layouts spezifiziert man auf zwei verschiedene Arten:
\begin{itemize}
	\item \textbf{Statisch / Deklarativ (XML)}
	\item \textit{Grundsätzlich in MOBPRO verwendet, bietet viele Vorteile (Deklarativ, weniger umständlich als Code, Struktur eminent, Umformungen ohne Rekompilierung möglich...)}
		\begin{itemize}
			\item Deklarative Beschreibung des GUI als Komponentenbaum
			\item XML-Datei unter \textit{res/layout}
			\item Referenzen auf Bilder/Texte/etc.
			\item Typischerweise ein XML pro Activity
		\end{itemize}
	\item \textbf{Dynamisch (in Java)}
	\item \textit{Jedes XML hat eine korrespondierende Java-Klasse, XML $\rightarrow$ Java = Inflating}
		\begin{itemize}
			\item Aufbau und Definition des GUI im Java-Code
			\item Normalerweise nicht nötig: die meisten GUIs haben fixe Struktur
			\item Änderung von Eigenschaften während Laufzeit ist normal (Bsp. Visibility, Ausblenden einer View, wenn nicht benötigt)
		\end{itemize}
\end{itemize}
\newpage
\subsection{XML-Layout}
\begin{itemize}
	\item Jedes Layout ist ein eigenes XML-File
	\begin{itemize}
		\item Root-Element = View oder ViewGroup
		\item Kann Standard- oder eigene View-Klassen enthalten
	\end{itemize}
	\item XML können mit Inflater "aufgeblasen" bzw. instanziiert werden, damit eigene wiederverwendbare Komponenten/Templates/Prototypen erzeugt werden können
	\item Innere Elemente können unterhalb eines Parents via View-ID referenziert werden (\textit{findViewById()})
	\item Debugging mit dem Layout-Inspector	
\end{itemize}

\subsubsection{Constraint-Layout}

\begin{multicols}{2}
\begin{itemize}
	\item Erstellung von komplexen Layouts, ohne zu schachteln
	\item Elemente werden relativ mit Bedingungen platziert
	\begin{itemize}
		\item zu anderen Elementen
		\item zum Parent-Container
		\item Element-Chains (spread/pack)
	\end{itemize}
	\item Layout-Hilfen (Hilfslinien, Barriers)
\end{itemize}
\columnbreak
\begin{minipage}[c]{\columnwidth}
	\centering
	\includegraphics[width=0.7\linewidth]{img/constraintlayout.jpg}
	\captionof{figure}{Constraint Layout}
\end{minipage}
\end{multicols}

\subsubsection{LinearLayout}

\begin{multicols}{2}
\begin{itemize}
	\item Reiht Elemente neben-/untereinander auf
	\begin{itemize}
		\item kann geschachtelt werden, um Zeilen/Spalten zu formen (nicht zu tief, sonst schlechte Performanz
	\end{itemize}
	\item Eigenschaften:\\
	 \textit{(orientation, gravity, weigthSum, etc.)}
	\item Layout-Parameter für Children
	\begin{itemize}
		\item layout\_width, layout\_height
		\item layout\_margin...
		\item layout\_weight, layout\_gravity
	\end{itemize}
\end{itemize}
\columnbreak
\begin{minipage}[c]{\columnwidth}
	\centering
	\includegraphics[width=0.5\linewidth]{img/linearlayout.jpg}
	\captionof{figure}{LinearLayout}
\end{minipage}
\end{multicols}

\paragraph{Warum nutzt man trotzdem noch LinearLayout?}
\begin{itemize}
	\item Nach wie vor einfachste Lösung für Button- oder Action-Bars ("flow semantik") und einfache Screens
	\item Kaum Konfiguration nötig, robust
	\item Für scrollbare Listen mit dynamischer Anzahl Elemente besser \textit{ListView} verwenden (siehe Adapter-Views)
	\item Einsatz mit Bedacht durchaus sinnvoll
\end{itemize}
\vspace{1em}
\noindent
Es gibt noch die \textbf{ScrollView}, deren Nutzung vertikales Scrollen bei zu grossen Layouts erlaubt, sie kann jedoch \underline{nur ein Kind} haben und enthält typischerweise das Top-Level-Layout einer Bildschirmseite.

\paragraph{Pixalangaben}
\textit{(Typischerweise werden Angaben in dp verwendet, ausser sp bei Schriftgrössen.)}
\begin{itemize}
	\item \textbf{dp - density-independent:} \\ Passen sich der physischen Dichte des Screens an, dp passen sich gegenüber den realen Dimensionen eines Screens und dessen Verhältnisse an.
	\item \textbf{sp - scale-independent:} \\Ähnlich der dp-Einheit, passt sich jedoch der Schriftskalierung des Nutzers an.
	\item \textbf{px - Pixels:} \\Passen sich der Anzahl Pixel eines Bildschirms an, deren Nutzung wird nicht empfohlen.
\end{itemize}

\newpage

\subsection{Ressourcen, Konfigurationen und Internationalisierung}
\textbf{Ressourcen} sind alle Nicht-Java-Teile einer Applikation und sind im \textit{/res}-Verzeichnis abgelegt, sogennante ausgelagerte Konstanten-Definitionen. Sie werden im Layout und Java-Code über die \textbf{\textit{automatisch generierte R-Klasse}} mit ID-Konstanten (int) referenziert. Kontextabhängige Ressourcen sind möglich z.Bsp. für Sprache, Gerätetyp, Orientierung, ...\\
\textbf{Beispiele}: Strings, Styles, Colors, Dimensionen, Bilder (drawables), Layouts (portrait, landscape), Array-Werte (z.Bsp. für Spinner) und Menü-Items\\

\begin{figure}[htb!]
	\centering
	\begin{subfigure}{0.5\textwidth}
		\centering
		\includegraphics[width=0.8\linewidth]{img/references_xml.jpg}
		\caption{Referenz in XML mit @}
	\end{subfigure}%
	\begin{subfigure}{0.5\textwidth}
		\centering
		\includegraphics[width=0.8\linewidth]{img/references_code.jpg}
		\caption{Referenz in Code über \texttt{R}-Klasse, diese wird beim Build automatisch generiert}
	\end{subfigure}
\end{figure}
\noindent
Für verschiedene Systemkonfigurationen benötigt es unterschiedliche Ausprägungen einer Ressource, beispielsweise:
\begin{itemize}
	\item \textbf{Internationalisierung}: komplette/teilweise Übersetzung, für diese werden unterschiedliche Ordner je nach Land/Sprache und seperate .xml angelegt
	\item \textbf{Auflösungsklassen}: \texttt{ldpi} (~120dpi), \texttt{mdpi} (~160dpi), \texttt{hdpi} (~240dpi), \texttt{xhdpi} (~320dpi) 
	\item \textbf{Orientierung} des Displays: landscape / portrait
	\item Verschiedene \textbf{HW-Modelle}: HTC, Samsung, Sony, LG, ...
\end{itemize}
Default-Verzeichnisse sind innerhalb von \texttt{res/} angelegt: drawable, layout, menu, values, ...\\
Bei spezifischen Konfigurationen werden meist Kopien der Default-Verzeichnisse/Ordner mit einem Suffix angelegt, bspw. \texttt{res/strings-de-rCH}, in welchen dann die Ressourcen (XML) erneut angelegt werden.
\begin{figure}[htb!]
	\centering
	\includegraphics[width=0.3\linewidth]{img/ressources.jpg}
	\caption{Beispiel der Default-Ressourcen}
\end{figure}

\newpage
\subsection{UI-Event-Handling}
\begin{itemize}
	\item Jedes View-Element hat eine entsprechende Java-Klasse (auch View-Groups!) \\$\rightarrow$ Layout könnte auch dynamisch in Java programmiert werden
	\item APIs der einzelnen View-Klassen sind \href{http://developer.android.com/reference/android/widget/package-summary.html}{hier} oder unter "Nützliche Links" genauer beschrieben
\end{itemize}
\begin{figure}[htb!]
	\centering
	\includegraphics[width=0.5\linewidth]{img/layout_id.jpg}
	\includegraphics[width=0.5\linewidth]{img/reference_code.jpg}
	\caption{ID im Layout erfassen und Referenz im Code}
\end{figure}

\subsubsection{GUI-Events}
\begin{itemize}
	\item \textbf{Observer/Listener}: einen Listener für ein entsprechendes Event bei der View registrieren, bspw. bei \texttt{Button myButton}:\\
	\texttt{myButton.setOnClickListener(listener)}
	\item verschiedenste Event- und Listener-Typen:\\
	\texttt{OnClickListener,  OnLongClickListener,  OnKeyListener,  OnTouchListener,  OnDragListener, ...} 
	\\$\rightarrow$ \texttt{public static} Interfaces der Klasse \texttt{View}
\end{itemize}
\textbf{Ziel:} Auf Klick-Event eines Buttons reagieren
\begin{itemize}
	\item Button muss eine ID haben im layout.xml
	\item Registrierungs eines Listeners an die View (Button) im Code:
\end{itemize}
\begin{lstlisting}
Button button = (Button) findViewById(R.id.question_button_done);
button.setOnClickListener(new OnClickListener() {
	@Override
	public void onClick(View v) {
		// handler code
		buttonClicked();
	}
});
\end{lstlisting}
\textbf{onClick-Event-Registrierung in XML}
\begin{figure}[htb!]
	\centering
	\includegraphics[width=0.6\linewidth]{img/onclick.jpg}
	\caption{Definition onClick-Handler im Layout $\rightarrow$ so nur für OnClick-Events}
\end{figure}
\begin{lstlisting}
// Implementierung OnClick-Handler-Methode in der Activity
public void increaseInternalCounter(View button){
	// ... handler code ...
}
\end{lstlisting}

\newpage

\subsubsection{Exkurs: Data Binding}

\begin{figure}[htb!]
	\centering
	\includegraphics[width=0.5\textwidth]{img/data_binding_model.png}
	\caption{Modell für Data Binding}
\end{figure}
\noindent
\textbf{Data Binding:} separiert UI und Daten, synchronisiert UI mit Daten (1-, resp. 2-way-binding), verwendet «binding expressions» mit @{..} Syntax im Layout-File, um View-Attribute zu initialisieren. Anbei ein Beispiel (auskommentiert):

\begin{lstlisting}
<?xml version="1.0" encoding="utf-8"?> 
<layout xmlns:android="http://schemas.android.com/apk/res/android"> 
	<data> 
		<variable name="model" type="org.example.MyModel"/> 
	</data>  // Definition der Layout-Variablen
	<LinearLayout ...> 
		<Button 
			android:id="@+id/button" 
			...
			android:enabled="@{model.user.role == `admin`}" 
			android:text="@{model.buttonText}" // Data Binding (1-way)
			...
			android:onClick="@{() -> model.increaseClickCount()}" /> // Event Binding
		<EditText 
			android:id="@+id/input"
			...
			android:text="@={model.inputText}"/> // Data Binding (2-way)
	</LinearLayout> 
</layout>

protected void onCreate(Bundle savedInstanceState) { 
	super.onCreate(savedInstanceState); 
	ActivityMainBinding binding = DataBindingUtil.setContentView(...); 
	model = new MainModel();
	model.load(); 
	binding.setModel(model);
	// Binden der Layout-Daten auf effektive Daten
	// z.B. ViewModel mit Observables
}
\end{lstlisting}
\newpage

\subsection{Options-Menü}

\begin{itemize}
	\item Android-Apps können oben rechts ein Menü mit Optionen anbieten
	\item Erzeugung durch Aufruf \textit{Hook} in der Activity-Klasse:\\
	\texttt{onCreateOptionsMenu(Menu menu)}
	\begin{itemize}
		\item Hier kann ein Menü mit Einträgen bestückt werden
		\item \texttt{MenuInflater} + XML benutzen oder Java oder beides
	\end{itemize}
	\item Beim Klick auf Eintrag Aufruf eines anderen Hooks:\\
	\texttt{onOptionsItemSelected(MenuItem item)}
\end{itemize}
\noindent
Für ein Options-Menü muss eine .xml-Datei (Bsp. main\_menu.xml) im Ordner \texttt{res/menu} angelegt werden. Danach werden Informationen folgendermassen eingetragen:
\begin{figure}[htb!]
	\begin{subfigure}{0.5\textwidth}
		\centering
		\includegraphics[width=\linewidth]{img/menu_example.jpg}
		\caption{Menü und Items in XML definieren}
	\end{subfigure}
	\begin{subfigure}{0.5\textwidth}
		\centering
		\includegraphics[width=\linewidth]{img/menuinflater.jpg}
		\caption{Menü mit \texttt{MenuInflater} aufblasen}
	\end{subfigure}
\end{figure}

Um bspw. einen String in einem Menüpunkt einzufügen, gibt es drei verschiedene Möglichkeiten:
\begin{figure}[htb!]
	\centering
	\includegraphics[width=.8\textwidth]{img/menu_string.jpg}
	\caption{Möglichkeiten zum Einlesen eines Strings}
\end{figure}
\begin{figure}[htb!]
	\centering
	\includegraphics[width=.7\textwidth]{img/menu_selektierung.jpg}
	\caption{Event-Handling: Selektierung}
\end{figure}

\newpage

\subsection{Adapter-Views}
\emph{Behandelt wird hier nur das synchrone Laden von kleinen/schnellen Datenquellen, für asynchrones Laden von langsamen/grossen Datenquellen konsultiere Doku über \textbf{Loaders}.}
\begin{figure}[htb!]
	\centering
	\includegraphics[width=.8\textwidth]{img/adapter.png}
	\caption{Aufgabe des Adapters}
\end{figure}
\begin{itemize}
	\item Adapter $\rightarrow$ Verbindung zwischen Datenquelle und GUI
	\item Zapft \textit{Datenquelle} an und beliefert \textit{AdapterView}
	\item Erzeugt (Sub-)Views pro gefundenes Datenelement
	\item Transformiert Daten ggf. in benötigtes Zielformat
	\item Datenquellen:\\
	String-Array, String-Liste, Bilder, Datenbank, ...
\end{itemize}

\vspace{3em}

\begin{figure}[htb!]
	\centering
	\includegraphics[width=.8\textwidth]{img/adapter_string.png}
	\caption{Beispiel eines ArrayAdapter}
\end{figure}
\begin{itemize}
	\item Bindet irgend ein Array oder Liste mit beliebig getypeten Elementen an irgend eine AdapterView
	\item Für jedes Daten-Element wird eine SubView erzeugt
	\item \textbf{Default}: Erstellt \texttt{TextView} mit \texttt{element.toString()}-Wert
\end{itemize}
\begin{figure}[htb!]
	\centering
	\includegraphics[width=.8\textwidth]{img/adapterview_example.jpg}
	\caption{Beispiel einer AdapterView}
\end{figure}

\subsubsection{AdapterViews \& ListActivity}

\begin{itemize}
	\item \textbf{AdapterViews}: spezielle View-Klassen
		\begin{itemize}
			\item Sind für Zusammenarbeit mit Adaptern optimiert \\
			(Bsp.\texttt{ListView, GridView, Gallery, Spinner, Stack,...})
			\item Füllen Teile von sich mit von Adaptern erzeugten Views
			\item Leiten ab von \texttt{android.widget.AdapterView<T extends android.widget.Adapter>}
		\end{itemize}
	\item Spezielle Activity: \textbf{ListActivity}
		\begin{itemize}
			\item Vordefiniertes Layout (enthält eine \texttt{ListView}, kein XML nötig)
			\item Vordefinierte Callbacks (bei Auswahl einer List-Entry)
			\item Bietet Zugriff auf aktuelle Selektion / Datenposition
		\end{itemize}
\end{itemize}

\newpage

\subsubsection{android.widget.Spinner}

\begin{itemize}
	\item ComboBox oder DropDown-List genannt (weitere Alternative: AutoCompleteTextView)
	\item Zeigt ein ausgewähltes Element, bei Klick erscheint ein Auswahlmenü
	\item 2 Varianten, um Daten auf Spinner zu setzen:
		\begin{itemize}
			\item Im Code mit Adapter:\\
			\texttt{spinner.setAdapter(myAdapter)}
			\item Im XML mit Angabe einer String-Array-ID:\\
			\texttt{android:entries="@array/spinnerValues"}
		\end{itemize}
	\item Listener setzen für Behandlung der Auswahl:\\
	\texttt{spinner.setOnItemSelectedListener(...)}
\end{itemize}
\vspace{2em}
\begin{figure}[htb!]
	\centering
	\includegraphics[width=\textwidth]{img/spinner_demo.png}
	\caption{Übungs-Demo aus der Vorlesung SW02 - Spinner}
\end{figure}

\newpage

\subsubsection{android.widget.ListView}

\begin{itemize}
	\item Liste von Views/Items, die zur Auswahl stehen
	\item Braucht viel Platz! Meist wird ihr der ganze Bildschirm zugeteilt
	\item i.d.R. zusammen mit \texttt{ListActivity} verwendet, Verwendung:
		\begin{enumerate}
			\item Navigiere zu eigener ListActivity
			\item Auswahl $\rightarrow$ Resultat setzen $\rightarrow$ finish
			\item Auswertung des Rückgabewert im Caller
		\end{enumerate}
	\item Konzeptionell identisch zum Spinner, jedoch andere Darstellung auf UI
		\begin{itemize}
			\item Verwendungsentscheid:
			\begin{itemize}
				\item Kurze Liste $\rightarrow$ Spinner
				\item (Sehr) lange Listen $\rightarrow$ ListView / ListActivity
				\item Kennt der User die möglichen Auswahlwerte $\rightarrow$ AutoCompleteTextView
			\end{itemize}
			\item Adapter- / Datendefinition grundsätzlich bei beiden gleich\\
			(d.h. im Code oder durch XML-Array)
			\item Auswahlmodus: \texttt{setChoiceMode(ListView.CHOICE\_MODE\_*} \\
			$\rightarrow$ Single- / Multiselection
		\end{itemize}
\end{itemize}

\subsubsection{android.app.ListActivity}

\begin{itemize}
	\item Spezielle Activity zur Darstellung einer ListView
	\item Vordefiniertes Layout (full-screen Liste)
		\begin{itemize}
			\item \texttt{setContentView(...)} muss nicht aufgerufen werden
			\item Aufruf i.d.R. mit \texttt{startActivityForResult(...)}
			\item Vordefinierte vererbte Konfigurationsmethoden
				\begin{itemize}
					\item \texttt{setListAdapter(adapter)} setzt Daten für die Liste
					\item \texttt{getListView()} erlaubt Zugriff auf ListView-Instanz\\
					\textit{(anstelle von findViewById(..) + Casten)}
				\end{itemize}
		\end{itemize}
	\item Callback bei der Auswahl
		\begin{itemize}
			\item \texttt{onListItemClick(parentView, view, position, id)}\\
			Wird bei Auswahl aufgerufen (muss in Subklasse überschrieben werden, keine Listener-Registrierung nötig)
		\end{itemize}
\end{itemize}

\begin{figure}[htb!]
	\centering
	\includegraphics[width=.8\linewidth]{img/listview_demo.png}
	\caption{Übungs-Demo aus der Vorlesung SW02 - ListView / ListActivity}
\end{figure}

\newpage

\subsection{ViewModel - Konfigurationswechsel \& temporäre Datenspeicherung}

Bei jedem Konfigurationswechsel (z.B. Änderung Bildschirmorientation) wird die aktuelle Activity-Instanz zerstört und neu aufgebaut. Dabei besteht das Problem des \textbf{Zustandsverlusts}. Der Zustand alles Views mit einer ID (mit einigen Ausnahmen) wird automatisch gesichert und wiederhergestellt. Der \textbf{inhärente Zustand}, alles was nicht sichtbar und in Feldern gespeichert ist, geht jedoch verloren. Um entgegenzuwirken, kann ein \textbf{ViewModel} verwendet werden.

\begin{figure}[htb!]
	\centering
	\includegraphics[width=.6\textwidth]{img/viewmodel}
	\caption{Position des ViewModels in der temp. Datenspeicherung}
\end{figure}

\begin{itemize}
	\item Kapselt UI-Daten so, dass diese bei einer Konfigurationsänderung einer Activity in-memory erhalten bleiben \textit{(Für den Fall eines App-Kills müssen Daten immer noch persistiert werden)}
	\item Lebensdauer mit der Activity gekoppelt
	\item Weniger Aufwand für Behandlung von Konfigurationsänderungen
\end{itemize}

\begin{figure}[htb!]
	\centering
	\includegraphics[width=\textwidth]{img/viewmodel_demo.png}
	\caption{Übungs-Demo aus der Vorlesung SW02 - ViewModel}
\end{figure}

\newpage

\subsection{Rückmeldungen an den Benutzer}

\subsubsection{Toast}

\begin{itemize}
	\item Kurze Rückmeldung (Popup) an den Benutzer, keine Interaktion möglich, verschwindet nach gewisser Zeit.
	\item Konfiguration: Text, Layout, Anzeigezeit (kurz/lang), Ort (gravity)
	\item Toasts mit eigenem Layout werden mit CustomToastView erstellt
\end{itemize}
\vspace{1em}
Beispielcode zur Erstellung von Toasts:

\begin{lstlisting}
// Default-Toast: Einzeiler
Toast.makeToast(getApplicationContext(), "Das ist..", Toast.LENGTH_LONG).show()
																						// LENGTH: Nur LONG oder SHORT
							// Kontext: meistens "this"
							
// Toast mit anderem Anzeigeort:
Context context = getApplicationContext();
Toast toast = Toast.makeText(context, "Toast links oben!", Toast.LENGTH_LONG);
toast.setGravity(Gravity.TOP|Gravity.START,0,0); // (x,y) Offset
toast.show()
\end{lstlisting}

\subsubsection{Alert-Dialog}

\begin{figure}[htb!]
	\centering
	\includegraphics[width=.35\textwidth]{img/alert_dialog.jpg}
	\caption{Beispiel eines Alert-Dialogs}
\end{figure}

\begin{itemize}
	\item Fenster mit Interaktionsmöglichkeiten für den Benutzer
		\begin{itemize}
			\item Information / Eingabe von Daten
			\item Interaktion möglich
			\item Buttons: positive, neutral, negative
		\end{itemize}
	\item Vorteile
	\begin{itemize}
		\item Kaum Einschränken in punkto Darstellung
		\item Vorbereitet für die Anzeige von Daten
		\item Verschwindet erst, wenn sie vom Benutzer quittiert wurde
	\end{itemize}
	\item Konfiguration: Buttons, Titel, Icon, Nachricht\\
	Inhalt: Liste von Items oder eigene View
\end{itemize}

\newpage

\begin{itemize}
	\item Vorgehen beim Erstellen eines Alert-Dialog mit Builder-Muster
	\begin{enumerate}
		\item Builder erstellen: \texttt{new AlertBuilder.Builder(this)}
		\item Builder konfigurieren: \\
		\texttt{setXXX} + Registrierung von \texttt{ClickListeners}
		\item Dialog erstellen: \texttt{Dialog dialog = builder.create()}
		\item Dialog anzeigen: \texttt{dialog.show()}
	\end{enumerate}
	\item Anzeige von Dialogen ist \textbf{immer asynchron!}\\
	Bei \texttt{show()} wird nicht gewartet, kein Rückgabewert\\
	$\rightarrow$ Behandlung von Benutzerselektion mit Listener
\end{itemize}

\begin{figure}[htb!]
	\centering
	\includegraphics[width=.9\textwidth]{img/alertdialog.jpg}
	\caption{Beispiel eines AlertDialog aus Vorlesung}
\end{figure}

\newpage

\begin{figure}[htb!]
	\centering
	\includegraphics[width=\textwidth]{img/alertdialog_daten.png}
	\caption{Beispiel mit Auswahl-Daten}
\end{figure}
\begin{figure}[htb!]
	\centering
	\includegraphics[width=\textwidth]{img/alertdialog_custom.png}
	\caption{Beispiel mit eigenem Layout}
\end{figure}

\newpage
\noindent
Ein (offener) Dialog gehört zum Zustand einer Activity, ist ein Dialog noch geöffnet bei einem Konfigurationswechsel, dann wird dieser nicht gespeichert und auch nicht wiederhergestellt! Deshalb sollten Dialoge als \texttt{DialogFragment} implementiert werden. Der Zustand des Dialogs wird dann vom \texttt{FragmentManager} korrekt mit Lifecycle und Activity synchronisiert (save/restore)\\
Für den Moment: Ein \textbf{Fragment} ist ein wiederverwendbarer "UI Schnippsel" mit eigenem Zustand und Lifecycle.

\subsubsection{Notifications (Status-Bar)}

\begin{itemize}
	\item Persistente Nachricht
	\begin{itemize}
		\item Kurze Ticker-Nachricht in der Status-Bar
		\item Danach persistente Anzeige im Notification Window
		\item Bei Auswahl erfolgt Aufruf einer definierten Activity
	\end{itemize}
	\item Vorteile:\\
	Nachricht bleibt erhalten bis vom Nutzer quittiert\\
	Beliebig komplexe Behandlung, da Start einer Activity
	\item Nachteil:\\
	Etwas komplexere Mechanik wegen \texttt{PendingIntent}
\end{itemize}

\begin{figure}[htb!]
	\centering
	\includegraphics[width=\textwidth]{img/notification_demo.png}
	\caption{Übungs-Demo aus der Vorlesung SW02 - Notification}
\end{figure}

\newpage
\section{Android 3 - Persistenz \& Content Providers}

Persistenz: Daten über Laufzeit der App erhalten.
Für lokale Persistenz gibt es drei Möglichkeiten:
\begin{itemize}
	\item \textbf{Shared Preferences}\\
	Key/Value-Paare, Verwendung für kleine Datenmengen
	\item \textbf{Dateisystem}\\
	intern oder extern, in App-Sandbox (privat) oder auf SD-Karte (öffentlich), Verwendung für binäre/grosse Dateien, Export
	\item \textbf{Datenbank (Room)}\\
	SQLite + Object Relational Mapper (ORM), Verwendung für strukturierte Daten + Abfragen/Suche
\end{itemize}

\subsection{(Shared) Preferences}
\begin{itemize}
	\item Jede Activity hat ein SharedPreferences-Profil, persistente Einstellungen für Activity oder Applikation
	\item Key-Value-Store (persistente Map)
	\item Preferences für \textbf{Activity}: \\
	\texttt{Activity.getPreferences(mode)}\\
	Anwendungsfall: Activity-State persistent speichern
	\item Preferences für \textbf{Applikation}: \\
	\texttt{PreferenceManager.getDefaultSharedPreferences(ctx)}\\
	\texttt{Context.getSharedPreferences(name, mode)}
	\item Mögliche Datentypen für Preferences-Werte:\\
	\textit{String, int, float, long, boolean, Set<String>} (mit seperaten Werten)
\end{itemize}
\paragraph{Lesen und Schreiben auf Preferences}
\begin{itemize}
	\item Mehrere Dateien pro Applikation möglich, Zugriff mit\\
		\texttt{Activity.getSharedPreferences(name, mode)} (unterschiedliche Dateinamen)\\
		oder auch über \texttt{getDefaultSharedPreferences(mode)}, die Applikation findet danach anhand der Preference-Benennungen die Einträge auch selber
	\item Lesen mit Methoden \texttt{SharedPreferences.get\textbf{X}()}\\
		\textbf{X} steht für den Typ, also String, Int, Boolean, ...
	\item Schreiben immer mit dem Editor:
		\begin{enumerate}
			\item \texttt{SharedPreferences.Editor editor = preferences.edit()}
			\item \texttt{editor.put\textbf{X}(...)}
			\item \texttt{editor.apply()} Persistierung der Änderungen
				\begin{itemize}
					\item asynchrone Persistierung, blockiert die Methode nicht
					\item für synchrone Persistierung: \texttt{editor.commit()}
				\end{itemize}
		\end{enumerate}
\end{itemize}
\vspace{1em}
Beispiel, um die Anzahl Aufrufe einer App über die Lebenszeit der App hinaus zu persistieren:

\begin{lstlisting}
final SharedPreferences preferences = getPreferences(MODE_PRIVATE);
final int newResumeCount = preferences.getInt(COUNTER_KEY, 0) + 1;
final SharedPreferences.Editor editor = preferences.edit();
editor.putInt(COUNTER_KEY, newResumeCount);
editor.apply();
\end{lstlisting}

\subsubsection{Darstellung User-Preferences}

	\begin{itemize}
		\item Automatische Darstellung mit \texttt{PreferenceFragment}, eigener Editor für jeden Wertetyp
		\item \texttt{PreferenceFragment} schreibt/liest grundsätzlich in die DefaultSharedPreferences, kann aber auch für andere Preference-Stores konfiguriert werden
	\end{itemize}

\newpage
\noindent
User-Präferenzen können in XML deklariert werden unter \texttt{res/xml} z.Bsp. als \texttt{preferences.xml}, wobei untersch. Präferenzen bspw. als \texttt{CheckBoxPreference, ListPreference} usw. erfasst werden. Daten können wie in diesem Beispiel aus den Array-Ressourcen bezogen werden:

\begin{figure}[htb!]
	\centering
	\includegraphics[width=.7\textwidth]{img/prefs_xml.jpg}
	\caption{Beispiel eines Präferenzen-XML}
\end{figure}

\begin{itemize}
	\item Ohne \texttt{android:summary} würde die gewählte Preference angezeigt werden
	\item \texttt{android:dependency} deklariert eine Abhängigkeit zu einer anderen Preference, ist diese nicht gegeben kann die andere Preference nicht ausgewählt werden
	\item \textbf{Entries}: "Anzeigestring", übersetzbar\\
	\textbf{EntryValues}: "Werte", nicht übersetzt, technischer Schlüssel
\end{itemize}
\begin{lstlisting}
// Zur "Uebersetzung" von Values zu Entries (Beispiel)
public String getValueFromKey(String key) {
	String[] keys = getResources().getStringArray(R.array.teaSweetenerValues);
	String[] values = getResources().getStringArray(R.array.teaSweetener);
	int i = 0;
	while(i < keys.length) {
		if(keys[i].equals(key)) {
			return values[i];
			}
		i++;
	}
	return "";
}
\end{lstlisting}

\newpage

\subsubsection{PreferenceFragment}

Ein \texttt{PreferenceFragment} kann in einer eigenen Activity (hier \texttt{TeaPreferenceActivity}) erstellt werden:

\begin{lstlisting}
public class TeaPreferenceActivity extends Activity {

	@Override
	protected void onCreate(Bundle savedInstanceState) {
		super.onCreate(savedInstanceState);
		getFragmentManager().beginTransaction().replace(android.R.id.content,
			new TeaPreferenceInitializer()).commit();
	}
	
	// PreferenceFragment als statische innere Klasse
	public static final class TeaPreferenceInitializer extends PreferenceFragment {
		@Override
		public void onCreate(final Bundle savedInstanceState) {
			super.onCreate(savedInstanceState);
			addPreferencesFromResource(R.xml.preferences);
			// Referenz auf preferences.xml
	}
}
\end{lstlisting}

\subsubsection{Default-Präferenzen}

Präferenzen können programmatisch auch wieder auf "Standard"-Werte oder auf festgelegte Werte gesetzt werden, für das Tee-Beispiel kann dies bspw. folgendermassen vorgenommen werden:

\begin{lstlisting}
SharedPreferences teaPrefs = PreferenceManager.getDefaultSharedPreferences(this);
SharedPreferences.Editor editor = teaPrefs.edit();
editor.putString("teaPreferred", "Lipton/Pfefferminztee");
editor.putString("teaSweetener", "natural");
editor.putBoolean("teaWithSugar", true);
editor.apply();
\end{lstlisting}

\newpage
	
\subsection{Dateisystem}

\begin{itemize}
	\item \textbf{Einsatzbereiche}
		\begin{itemize}
			\item Speichern/Laden von binären Dateien (Bilder, Musik, Video, Java-Objects, etc.)
			\item Caching (Heruntergeladene Dateien)
			\item Grosse Text-Dateien(Plain Text, Strukturierte Daten wie XML, JSON,))
		\end{itemize}
	\item Teilen / Freigeben von erstelltem Inhalt (Externer Speicher wie SD-Karte)
\end{itemize}

\begin{itemize}
	\item Dateien sind entweder
	\begin{itemize}
		\item PRIVATE $\rightarrow$ ins Applikationsverzeichnis\\
		\textit{(Zugriff für andere Apps nur über Content Provider möglich)}
		\begin{itemize}
			\item \texttt{Context.getFilesDir()}
		\end{itemize}
		\item PUBLIC $\rightarrow$ auf die SD-Karte 
		\begin{itemize}
			\item \texttt{Environment.getExternalStorageDirectory()}\\
			\texttt{Environment.getExternalStorageState();}
		\end{itemize}
	\end{itemize}
	\item Für Zugriff auf SD-Karte muss die Permission im Manifest eingetragen werden! (siehe nachfolgend)
\end{itemize}
	
\subsubsection{Exkurs: Permission-Model}

	\begin{itemize}
		\item Vor gewissen Operationen müssen Apps die Berechtigung des Nutzers erhalten (Kontaktzugriff, Internet, SD-Karte, Kamera, SMS, etc.)
		\item Klasse: \texttt{android.Manifest.permission}
		\item Seit API 23 werden keine dangerous Permissions mehr gewährt, der Nutzer muss diese selber freigeben (Applikation fragt beim Nutzer nach), Permissions werden einzeln gewährt/abgelehnt.\\
		\textbf{Konsequenz}: Apps müssen mit eingeschränkten Permissions umgehen können
		\item Arten von Permissions
		\begin{itemize}
			\item \textit{normal}
				\begin{itemize}
					\item Wird bei der Installation automatisch erlaubt
				\end{itemize}
			\item \textit{dangerous}
				\begin{itemize}
					\item Muss von User erlaubt werden (kann wieder entzogen werden)
				\end{itemize}
			\item \textit{signature}
				\begin{itemize}
					\item Wird automatisch erlaubt, wenn die App, welche die Permission definiert, vom gleichen Hersteller ist wie die App, welche die Permission beanträgt (sonst ist sie "dangerous")
				\end{itemize}
			\item \textit{signatureOrSystem}
				\begin{itemize}
					\item Wird automatisch erlaubt für Apps, welche im System-Image sind, sonst wie "signature"
				\end{itemize}
		\end{itemize}
		\item Permissions können gruppiert werden, User gibt Freigabe für alle Permissions in einer Gruppe (keine einzelnen Permissions), falls benötigt
	\end{itemize}
	
	\begin{figure}[htb!]
		\centering
		\includegraphics[width=.7\textwidth]{img/manifest_permissions.jpg}
		\caption{Erfassung von Permissions im Manifest}
	\end{figure}

\newpage

	\subsubsection{Exkurs ff: Runtime Permissions}
	
	\begin{figure}[htb!]
		\centering
		\includegraphics[width=.9\textwidth]{img/runtimeperm1.jpg}
		\caption{RuntimeCheck der Permissions}
	\end{figure}
	
	\begin{figure}[htb!]
		\centering
		\includegraphics[width=.9\textwidth]{img/runtimeperm2.jpg}
		\caption{Callback aus Permission-Abfrage}
	\end{figure}
	
	\subsubsection{Exkurs ff: Persistenz mit Datei}
	
	Repetition zu Streams, Reader \& Co.
	
	\begin{itemize}
		\item Stream: Byte-Datenstrom \texttt{[28,11,200,255,2,15,33]}
			\begin{itemize}
				\item Auf \texttt{File} öffnen:\\
				\texttt{FileOutputStream, FileInputStream}
			\end{itemize}
		\item Stream kann in Zeichenstrom \texttt{['h','a','l','l','o']} umgewandelt werden
			\begin{itemize}
				\item \texttt{FileReader, FileWriter} + "Buffered"-Versionen
			\end{itemize}
		\item Immer schliessen!\\
		\texttt{stream.close() / reader.close()}
		\item Nicht vergessen: \texttt{try-catch-finally} implementieren
		\item \texttt{java.nio.file.Path}: ist ab API 26 in Android verfügbar!
	\end{itemize}

	\begin{figure}[htb!]
		\centering
		\includegraphics[width=.7\textwidth]{img/persist_text.jpg}
		\caption{Beispielcode zur Persistierung in einem Textfile}
	\end{figure}

\newpage	
	
\subsection{Datenbank (Room)}

Android-DB \textbf{SQLite} ist bei Android fix integriert. Ein DB-Adapter ist die Verbindung zwischen Business-Objekten und einer Datenbank.

\begin{figure}[htb!]
	\centering
	\includegraphics[width=.85\textwidth]{img/sqlite_framework.png}
	\caption{SQLite Framework}
\end{figure}

\begin{itemize}
	\item Room ist ein Object Relational Mapper (ORM) für Android
		\begin{itemize}
			\item Klassen werden auf relationale DB-Tabellen gemappt
			\item Zugriff auf Datenbank wird abstrahiert\\
			$\rightarrow$ Typischerweise werden SQL-Statements durch Methodenaufrufe gekapselt
		\end{itemize}
	\item Spezialfälle des Room ORM
		\begin{itemize}
			\item \textit{Datenzugriff über DAO}: Queries werde als SQL-Statements in Annotationen definiert
			\item Beziehungen zwischen Entitäten müssen manuell abgebildet werden (Performance!)
			\item \textit{Nested Objects}: mehrere POJOs in einer Tabelle
			\item Einschränkungen für Datenzugriffe, standardmässig nicht möglich im UI Thread
		\end{itemize}
	\item Die drei Komponenten von Room
		\begin{description}
			\item[Database] Abstraktion der Datenverbindung
			\item[Entity] Repräsentation einer Tabelle in der relationalen DB
			\item[DAO] (Data Access Object) Enthält Methoden für Datenzugriff
		\end{description}
\end{itemize}

\begin{figure}[htb!]
	\centering
	\includegraphics[width=.5\textwidth]{img/room_components.jpg}
	\caption{Komponenten von Room}
\end{figure}

\newpage

\subsubsection{Room - Code-Beispiele}

\begin{lstlisting}
@Entity // POJO mit Annotationen
public class User {
	@PrimaryKey
	public int uid;
	
	@ColumnInfo(name = "first_name")
	public String firstName;
	
	@ColumnInfo(name = "last_name")
	public String lastName;
}
\end{lstlisting}

\begin{lstlisting}
@Dao // Datenzugriff ueber Annotationen (teilweise mit SQL-Queries)
public interface UserDao {
	@Query("SELECT * FROM user") 
	List<User> getAll(); 
	
	@Query("SELECT * FROM user WHERE uid IN (:userIds)") 
	List<User> loadAllByIds(int[] userIds); 
	
	@Insert 
	void insertAll(User... users); 
	
	@Delete 
	void delete(User user);
}
\end{lstlisting}

\begin{lstlisting}
// Database: Subklasse von RoomDatabase, konfiguriert mit Database Annotation
@Database(entities = {User.class}, version = 1)
public abstract class AppDatabase extends RoomDatabase {
	public abstract UserDao userDao();
}

// Zum Erzeugen einer Instanz der DB:
AppDatabase db = Room.databaseBuilder(
			getApplicationContext(),
			AppDatabase.class,
			"database-name"
).build();
\end{lstlisting}

\newpage

\subsubsection{Room - Daten mit Entitäten definieren}

\begin{itemize}
	\item Ein POJO mit \texttt{@Entity} Annotation
	\item Primärschlüssel (wird in jeder Entität benötigt)
		\begin{itemize}
			\item \texttt{@PrimaryKey} für ein einzelnes Feld, optional mit \texttt{autoGenerate} Property
			\item Für zusammengesetzte Primärschlüssel:\\
			\texttt{primaryKeys} Property in \texttt{@Entity} Annotation
		\end{itemize}
	\item Falls bestimmte Felder nicht gespeichert werden sollen
		\begin{itemize}
			\item \texttt{@Ignore} Annotation für ein einzelnes Feld
			\item mit \texttt{ignoredColumns} Property in \texttt{@Entity} Annotation für mehrere Felder \\
			(v.a. von Superklassen)
		\end{itemize}
\end{itemize}

\begin{lstlisting}
// Code-Beispiel
// ACHTUNG: dieses Beispiel definiert mehrere Primary Keys und vermischt Ansaetze zum Ignorieren von Feldern zwecks Syntax-Demonstration!

@Entity(primaryKeys = {"firstName", "lastName"},
			ignoredColumns = "password, otherField")
public class User extends Party {
	@PrimaryKey(autoGenerate = true)
	public int id;
	
	public String firstName;
	public String lastName;
	
	@Ignore
	Bitmap picture;
}
\end{lstlisting}

\subsubsection{Room - Beziehungen modellieren}

\textbf{Beachte:} Room erlaubt aus Performanzgründen keine Objektreferenzierungen!

\begin{lstlisting}
@Entity(foreignKeys = @ForeignKey(entity = User.class,
															parentColumns = "id",
															childColumns = "user_id"))
public class Book {
	@PrimaryKey
	public int bookId;
	
	public String title;
	
	@ColumnInfo(name = "user_id"
	public int userId; // Feld-Typ: nur ID, nicht User
}
\end{lstlisting}

\newpage

\subsection{Mit DAOs auf Daten zugreifen}

\begin{itemize}
	\item DAOs enthalten Methoden für den abstrahierenden Datenbankzugriff
	\item Trägt zur \textit{Separation of Concerns} bei und erhöht die Testbarkeit\\
	$\rightarrow$ DAOs können gemockt werden!
	\item DAOs als Interfaces oder abstrakte Klassen definieren\\
	$\rightarrow$ Room erzeugt passende Implementationen bei der Kompilierung\\
	\textit{(Typischerweise eine DAO-Klasse pro Entity, mit allen möglichen Operationen)}
	\item Zwei Möglichkeiten:\\
	Convenienve Queries \textbf{oder} \texttt{@Query} Annotation mit SQL-Statements
\end{itemize}

\subsubsection{Convenience Queries}

\begin{itemize}
	\item Werden über Annotations für die jeweiligen Methoden definiert:\\
	\texttt{@Insert, @Update, @Delete}
	\item Alle Parameter müssen Klassen mit einer \texttt{@Entity} Annotation \\
	(oder Collections/Arrays) davon sein
	\item Rückgabewerte:
		\begin{itemize}
			\item \textbf{Insert}: \texttt{long} bzw. \texttt{long[]} bzw. \texttt{List<Long>} \textit{(liefert Row-IDs zurück)}
			\item \textbf{Update / Delete}: \texttt{int} \textit{(Anzahl modifizierte Tabelleneinträge)}
		\end{itemize}
	\begin{lstlisting}
@Insert
public long[] insertUsersAndFriends(User user, List<User> friends);
		// ID Rueckgabe, sonst void
															// Parameter fuer Operation (Entities)
	\end{lstlisting}
\end{itemize}

\begin{lstlisting}
// Weitere Convenience Queries Beispiele

@Dao
public interface MyDao {

	@Insert (onConflict = OnConflictStrategy.REPLACE)
	public void insertUsers(User... users);
	@Insert
	public void insertBothUsers(User user1, User user2);
	@Insert
	public long[] insertUsersAndFriends(User user, List<User> friends);
	
	@Update
	public void updateUsers(User... users);
	
	@Delete
	public void deleteUsers(User... users);
}	
\end{lstlisting}

\newpage

\subsubsection{Custom Queries mit @Query}

\begin{itemize}
	\item DIe \texttt{@Query} Annotation kann für Schreib- und Lesevorgänge genutzt werden
	\item Jede \texttt{@Query} wird zur Kompilierzeit überprüft\\
	$\rightarrow$ Kompilierfehler bei ungültigen Queries, keine Laufzeitfehler!
	\item Für eine \texttt{@Query} kann eine beliebige Anzahl (0..n) Parameter verwendet werden
	\item Wenn nicht ganze Objekte benötigt werden, können Ressourcen gespart werden durch die Verwendung von POJOs mit \texttt{@ColumnInfo} Annotationen
\end{itemize}

\begin{lstlisting}
// Custom Queries Codebeispiele

@Dao 
public interface MyDao { 
	@Query("SELECT * FROM user") 
	public User[] loadAllUsers();

	@Query("SELECT * FROM user WHERE age > :minAge") 
	public User[] loadAllUsersOlderThan(int minAge);

	@Query("SELECT first_name, last_name FROM user WHERE region IN (:regions)") 
	public List<NameTuple> loadUsersFromRegions(List<String> regions); 
}

public class NameTuple { 
	@ColumnInfo(name = "first_name") 
	public String firstName;

	@ColumnInfo(name = "last_name") 
	public String lastName; 
}
\end{lstlisting}

\newpage

\subsection{DB-Einträge in einer Liste anzeigen}

\begin{itemize}
	\item Verschiedene Ansätze, je nach Umfang/Komplexität der Datensätze:\\
	ListView, RecyclerView, Kombination mit ViewModel und LiveData
	\item In jedem Fall werden spezifische Adapter benötigt, um die Daten auf Views zu mappen
\end{itemize}

\begin{figure}[htb!]
	\centering
	\includegraphics[width=\textwidth]{img/dbentries_list.png}
	\caption{Codebeispiel für das Darstellen von DB-Einträgen in einer Liste}
\end{figure}

\newpage

\subsection{Content Providers}

\begin{itemize}
	\item Content Provider stellen für andere Applikationen Daten bereit
	\item Die Daten stammen aus einer gekapselten DB \textbf{oder} aus dem privaten Dateisystem \textbf{oder} werden on-the-fly erzeugt
	\item Zugriff auf die Daten über URI (Uniform Ressource ID), Beispiel siehe in Abbildung \ref{fig:uri_content}
	\item Zwei Arten von URIs
		\begin{itemize}
			\item Pfad (Bezeichnete Datenmenge, vgl. Verzeichnit mit Daten)
			\item Item (Einzelnes Datenelement, vgl. einzelne Datei)
		\end{itemize}
\end{itemize}

\begin{figure}[htb!]
	\centering
	\includegraphics[width=.8\textwidth]{img/content_uri.png}
	\caption{Aufbau eines URI}
	\label{fig:uri_content}
\end{figure}
	
\subsubsection{Standard Content Providers}

\begin{itemize}
	\item Im Android-System gibt es bereits einige Content Providers, die genutzt werden können
		\begin{itemize}
			\item Kontakte: Namen, Telefon-Nummern, Emails, Adressen, etc.
			\item SMS/MMS: Erhaltene/Gesendete/Drafts SMS/MMS
			\item Media Store: Auf Gerät gespeicherte Audio-, Video-, Bilder-Daten
			\item Settings: Einstellungen für das Gerät
			\item Kalender: Kalender, Events, Erinnerungen, Teilnehmer, etc. 
		\end{itemize}
	\item Daten sind meist in mehreren Tabellen abgelegt
\end{itemize}	

\subsection{Exkurs : REST-ful Webservices}

\begin{itemize}
	\item Webservices auf der Basis von HTTP
	\item Grundidee (in purer Form)
		\begin{itemize}
			\item URL einer Ressourcensammlung \textit{(http://directory.com/contacts)}\\
			oder URL einzelner Ressource \textit{(http://directory.com/contacts/17)}
			\item HTTP-Methode = Operation auf Daten\\
			(GET, PUT, POST, DELETE)
			\item Antwort-Datenformat = XML, JSON, ...
		\end{itemize}
\end{itemize}

\begin{figure}[htb!]
	\centering
	\includegraphics[width=\textwidth]{img/state_transfer.png}
	\caption{Beispiel - Representational State Transfer}
\end{figure}

\newpage

\subsection{Content Resolver \& Content Provider}

\begin{itemize}
	\item Zugriff auf einen Content Provider erfolgt über einen \textbf{Content Resolver}\\
	\texttt{Context.getContentResolver()}
		\begin{itemize}
			\item Bietet DB-Methoden und Zugriff auf Content via Streams
				\begin{itemize}
					\item CRUD: \texttt{insert() / query() / update() / delete()}
					\item \texttt{openInputStream(uri) / openOutputStream(uri)}
				\end{itemize}
			\item Ein Content Resolver ist ein Proxy, der...
				\begin{itemize}
					\item ...URI auflöst und zuständigen Content Provider sucht / findet
					\item ...Interprozess-Kommunikation behandelt (aufrufende App ist meist in einem anderen Package als der aufgerufene Content Provider)
				\end{itemize}
		\end{itemize}
	\item Unter Umständen müssen die Permissions noch gesetzt werden (im Manifest)
\end{itemize}
	
\subsubsection{Zugriff auf Daten über Content Resolver \& Query}	

\begin{lstlisting}
Cursor cursor = getContentResolver().query( 
		contentUri, 			// The content URI of the table 
		projection, 			// The columns to return for each row 
		selectionClause, 	// Selection criteria 
		selectionArgs, 		// Selection criteria 
		sortOrder); 			// The sort order for the returned row
\end{lstlisting}
	
\begin{figure}[htb!]
	\centering
	\includegraphics[width=\textwidth]{img/compare_query.png}
	\caption{Vergleich: ContentProvider Query und SQL Query Parameter}
\end{figure}

\begin{figure}[htb!]
	\centering
	\includegraphics[width=.7\textwidth]{img/contentprovider.jpg}
	\caption{Content Provider - Anwendung (Data: Dateisystem, XML: Preferences)}
\end{figure}

\newpage

\begin{itemize}
	\item SMS des Systems sind über den Content Provider zugänglich\\
	\textit{(Benötigt Permission für SMS, diese testen und ggf. beantragen)}:
		\begin{itemize}
			\item \texttt{android.provider.Telephony.Sms}\\
			\textit{("Sub-Providers" für Sent, Inbox, Draft, etc.)}
			\item Im Package \texttt{android.provider.*} finden wir \\
			"Contract Klasse" \texttt{Telephone.Sms} mit \\
			Hilfsklassen \texttt{BaseColumns} und \texttt{Telephony.TextBasedSmsColumns}\\
			\textit{(Hier findet man Content-URI und Spalten-Namen für Projections)}
		\end{itemize}
\end{itemize}

\begin{figure}[htb!]
	\centering
	\includegraphics[width=\textwidth]{img/content_sms.jpg}
	\caption{Anwendungsbeispiel - Alle SMS mit Text anzeigen}
\end{figure}

Jeder Content Provider bietet eine eigene Standard-API, in der Android Dokumentation sind die Zugriffe auf Kontakte und Kalender gut dokumentiert (da dies eher komplizierte Modelle sind). Einstiegspunkt für die meisten Provider: \texttt{android.provider.*}

\subsection{Eigener Content Provider}

\begin{itemize}
	\item Um eigenen Content Provider zu schreiben, muss die eigene Klasse von der abstrakten Klasse\\ \texttt{android.content.ContentProvider} ableiten
	\item Wird bei App-Start hochgefahren und bleibt aktiv, in \texttt{onCreate()} kann eine Initialisierung vorgenommen werden (einzige Lifecycle-Methode)
	\item CRUD-Methoden: query, insert, update, delete (muss nicht alle implementieren)\\
	\textit{(Möglichkeit, einen read-only Content Provider anzulegen)}
\end{itemize}

\begin{figure}[htb!]
	\centering
	\includegraphics[width=.75\textwidth]{img/content_notes.png}
	\caption{Anwendungsbeispiel - Content Provider für Notizen}
\end{figure}

\section{Android 4 - Kommunikation \& Nebenläufigkeit}

	\subsection{Nebenläufigkeit}
	
		\subsubsection{Android und der Main-Thread}
	
		\begin{itemize}
			\item Eine Applikation baut ihr UI nur auf einem Thread, dem \textbf{main-Thread} auf\\
			$\rightarrow$ blockiert man den main-Thread, friert das ganze UI ein
			\item UI-Komponenten sind nicht Thread-safe\\
			$\rightarrow$ UI-Zugriff nur aus main-Thread, sonst Exception
			\item Netzwerk- und andere Methoden können lange dauern und sind blockierend, werden diese auf dem main-Thread aufgerufen, wird er blockiert und es werden keine UI-Events mehr aufgerufen\\
			Bsp. \texttt{URLConnection.connect(), Bitmap.resize(), Database.open(), ... }
		\end{itemize}
	
		\begin{figure}[htb!]
			\centering
			\includegraphics[width=.7\textwidth]{img/thread_sleep.jpg}
			\caption{Einfaches Blockieren der App mittels \texttt{Thread.sleep(long time)}}
		\end{figure}
	
		\subsubsection{Android-Überwachung - ANR (Application Not Responding)}
			\begin{itemize}
				\item Android überwacht Ansprechbarkeit von Apps nach gewissen Kriterien
					\begin{itemize}
						\item Keine Reaktion auf Input-Event innert 5 Sekunden
						\item Broadcast-Receiver nicht fertig innert 10 Sekunden
					\end{itemize}
				\item Möglicher Effekt ist ein ANR-Dialog ("App reagiert nicht")
				\item System-Mechanismus zum Stoppen "böser" Apps
				\item Fragen, die man sich stellt, um den main-Thread zu entlasten:
					\begin{itemize}
						\item Ist eine Hintergrundaufgabe aufschiebbar?
						\item Hat ein Task Auswirkungen auf das UI?
						\item Wartet der User auf ein Resultat?
					\end{itemize}
				\item Herausforderungen dabei:
					\begin{itemize}
						\item Das Resultat am Ende auf dem UI-Thread darstellen
						\item Ist das UI noch da?
					\end{itemize}
				\item Wenn eine Aktion nicht aufschiebbar ist:
					\begin{enumerate}
						\item \textbf{Klasse \texttt{AsyncTask}}\\
						Konstrukt zur Auslagerung zeitintensiver Aufgaben auf Background-Task, für die meisten Fälle ausreichend
						\item \textbf{Eigene \texttt{Thread }Instanzen}\\
						Standardimplementierung von Nebenläufigkeit in Java, kann komplex werden: Synchronisierung, Deadlocks, ...
						\item \textbf{Foreground Service}\\
						Hintergrundaktionen, die vom Nutzer bemerkt werden z.Bsp. Musik-Player
					\end{enumerate}
				\item Zurückgelangen zum UI-Thread:
					\begin{enumerate}
						\item Klasse \texttt{AsyncTask}: \\
						Spezielle Methoden auf Main-Thread, bspw. \texttt{onProgressUpdate, onPostExecute, ...}
						\item Eigener \texttt{Thread}, zwei Möglichkeiten:
							\begin{itemize}
								\item \texttt{Activity.runOnUiThread(Runnable action)}
								\item \texttt{View.post(Runnable action)}
								\item Klasse \texttt{android.os.Handler} $\rightarrow$ nutzt Message Queue von Thread
							\end{itemize}
					\end{enumerate}
			\end{itemize}
			\newpage
			\begin{figure}[htb!]
				\centering
				\includegraphics[width=.6\textwidth]{img/background_work.jpg}
				\caption{Verschiedene Arten für Hintergrundaufgaben}
			\end{figure}
		
			\paragraph{Langandauernde Operationen}
			Android lässt gewisse Operationen auf main-Thread gar nicht erst zu!
			Bsp. Netzwerk-API, bspw. Aufruf von \texttt{URLConnection.connect()} führt zu einer\\ \textbf{NetworkOnMainThreadException}\\
			Grund dafür ist, dass Netzwerk-Kommunikation lange dauern kann, Netzwerk-Calls nie auf UI-Thread ausführen, \textbf{AsyncTask oder eigenen Thread verwenden} (und UI-Aktualisierungen in Methoden auf dem main-Thread!)
	
	\newpage
	\subsection{Nebenläufigkeit: AsyncTask}
		\begin{figure}[htb!]
			\centering
			\includegraphics[width=\textwidth]{img/asynctask.png}
			\caption{\texttt{MyAsyncTask} Klasse}
		\end{figure}
		
		\begin{itemize}
			\item 3 Typ-Parameter
			\begin{itemize}
				\item \textbf{Params}: Typ der Input-Elemente, \\ Bsp. \texttt{URL} (Links auf Textfiles,...)
				\item \textbf{Progress}: Typ der Zwischenresultate, \\ Bsp. \texttt{String} (\texttt{Void} falls nicht genutzt) (Heruntergeladener Titel, ...)
				\item \textbf{Result}: Typ des Resultats, \\ Bsp. \texttt{Integer} (\texttt{Void} falls nicht genutzt) (Anzahl heruntergeladener Titel, ...)
			\end{itemize}
			\item 3 wichtige Methoden
				\begin{itemize}
					\item \texttt{doInBackground(Params...)}: Lange andauernd (Worker-Thread)
					\item \texttt{onProgressUpdate(Progress...)}: Zwischenresultat verarbeiten (UI-Thread)
					\item \texttt{onPostExecute(Result)}: Resultat verarbeiten (UI-Thread)
				\end{itemize}
			\item Schlussfolgerungen:
				\begin{enumerate}
					\item AsyncTasks werden standardmässig (seriell) durch einen einzelnen Worker-Thread ausgeführt.
					\item Thread-Pool (parallele Ausführung) durch Angabe Executor möglich.
					\item Implementierungs-Details (ggf. auch relevante) können sich ändern.
				\end{enumerate}
		\end{itemize}
		
		\begin{figure}[htb!]
			\centering
			\includegraphics[width=\textwidth]{img/ablauf_asynctask.png}
			\caption{Ablauf mit und ohne AsyncTask}
		\end{figure}
	
		\newpage
		\subsubsection{Einige Demos aus dem Unterricht}
		\begin{figure}[htb!]
			\centering
			\includegraphics[width=.9\textwidth]{img/async_demo_01.png}
		\end{figure}
		\begin{figure}[htb!]
			\centering
			\includegraphics[width=.9\textwidth]{img/async_demo_02.png}
		\end{figure}
	\newpage
		\begin{figure}[htb!]
			\centering
			\includegraphics[width=.9\textwidth]{img/async_demo_03.png}
		\end{figure}
		\begin{figure}[htb!]
			\centering
			\includegraphics[width=.9\textwidth]{img/async_demo_04.png}
		\end{figure}
	
	\newpage
	\subsection{Nebenläufigkeit: Threads}
		\paragraph{Repetition}
		\texttt{java.lang.Thread implements Runnable}
		\begin{itemize}
			\item Muss \texttt{Runnable} gesetzt haben, übergeben im Konstruktor, oder selber \texttt{run()} implementieren
			\item Wichtigste Methoden:
				\begin{itemize}
					\item \texttt{run()}
					\item \texttt{start()}
					\item \texttt{sleep (long)}
					\item \texttt{isAlive()}
				\end{itemize}
			\item Interface \texttt{java.lang.Runnable} (oder Lambda)
				\begin{itemize}
					\item Eine Methode: \texttt{run()}
					\item Implementierung des relevanten nebenläufigen Codes
				\end{itemize}
		\end{itemize}
	
		\begin{figure}[htb!]
			\centering
			\includegraphics[width=.9\textwidth]{img/threads_usage.png}
			\caption{Verwendung von Threads}
		\end{figure}
	
	
	\subsection{(Backend-) Kommunikation über HTTP}
	
		Apps können mit Server im Hintergrund kommunizieren (hält Daten, stellt Business-Logik bereit, authentisiert User etc.).
		Kommunikation findet i.d.R. über (REST) HTTP-API im Datenformat JSON (seltener XML) statt.
		
		\begin{itemize}
			\item HTTP: Hyper Text Transfer Protocol \\
					Zustandsloses Kommunikationsprotokoll
				
			\item Transport über TCP / IP
			\item Request / Response Muster (Anfrage/Antwort): \\
					GET, PUT, POST, DELETE
					
			\item Nachrichten bestehen aus Header \& Body
			\begin{itemize}
				\item 0 .. n Headers: Key-Value Paare
				
				\item Body (Content): beliebig, typischerweise Text
			\end{itemize}
			
			\item Mit jeder Antwort liefert Server einen Antwortcode (Bsp. 200 = OK)
			
		\end{itemize}
	
		\subsubsection{HTTP-Requests absetzen}
		
		\begin{itemize}
			\item \textbf{Veraltet:}\\
					\texttt{URL} und \texttt{URLConnection}: Standard-Java-Klassen, erlauben das Absetzen von HTTP-Requests, mühsame Verwendung, veraltete API
			
			\item \textbf{Besser:}\\
					HTTP-Client-Library verwenden, "headless Browser"\\
					Empfohlen: OkHttpClient, Gradle Dependencies: \\
					\texttt{com.squareup.okhttp3:okhttp:3.13.1}\\
					\texttt{com.squareup.okhttp3:logging-interceptor:3.12.1}
		\end{itemize}
	
		\begin{figure}[!htb]
			\centering
			\includegraphics[width=\textwidth]{img/okhttp_verwendung.png}
			\caption{Beispiel der Verwendung von OkHttpClient}
			\label{fig:okhttp_verwendung}
		\end{figure}
		
		\begin{itemize}
			\item Unverschlüsselte Kommunikation über HTTP seit Android 9 (API 28) unterbunden, kann im Manifest explizit erlaubt werden:\\
			\texttt{android:usesCleartextTraffic="true"}
			
			\item Für den Internetzugriff (und Netzwerkstatus) muss die Berechtigung vorliegen bzw. deklariert werden, dies kann im Manifest geschehen:\\
			\texttt{uses-permission android:name="android.permission.INTERNET"}\\
			\texttt{uses-permission android:name="android.permission.ACCESS\_NETWORK\_STATE"}
		\end{itemize}
		\vspace{1em}
		\textbf{Demo für lokales Testen der Backend-Kommunikation und Umwandlung von Binärdaten können in den MOBPRO-Folien \textit{"MobPro\_Android\_4"} eingesehen werden.}	
		
	\newpage	
	\subsection{JSON-Webservices mit Retrofit konsumieren}
	
	Möglichst immer REST-Semantik (HTTP-Methoden für gewünschte Operation) verwenden.\\
	
	\begin{itemize}
		\item \textbf{REST-ful Webservices}\\
				Webservice auf Basis von HTTP
				
		\item Grundidee:
		
		\begin{itemize}
			\item Base-URL: Ressourcensammlung / einzelne Ressource\\
					\textit{(http://directory.com/contacts/{17})}
			
			\item HTTP-Methode: Operation auf Daten\\
					\textit{(GET, PUT, POST, DELETE)}
			
			\item Antwort-Datenformat: XML, JSON, ...
		\end{itemize}
	\end{itemize}

	\begin{figure}[!htb]
		\centering
		\includegraphics[width=\textwidth]{img/restful_exmp.png}
		\caption{Beispiel für RESTful Webservice}
		\label{fig:restful_example}
	\end{figure}
		
	\subsubsection{JSON Parsing mit GSON}
	
	Gson ist ein JSON to Java Mapper, welcher JSON-Strukturen auf äquivalente Java-Klassen abbildet (ähnlich ORM bei Room).\\
	\textbf{Beispiel:}
	
	\begin{lstlisting}
String url = "http://www.nactem.ac.uk/.../dictionary.py?sf=HTTP"; 
OkHttpClient client = new OkHttpClient();
Request request = new Request.Builder().url(url).build(); 
Response response = client.newCall(request).execute(); 
String json = response.body().string(); 

Gson gson = new Gson(); 
Type listType = new TypeToken<List<AcronymDef>>(){}.getType(); 
List<AcronymDef> definitions = new Gson().fromJson(json, listType);
	\end{lstlisting}
	
	\subsubsection{Retrofit}
	
	Mit Retrofit möchte man das Backend als Interface abstrahieren, um HTTP-Calls mit Java-Calls zu ersetzen, wie folgt:
	
	\begin{lstlisting}
public interface AcronymService {

	@GET("dictionary.py")
	Call<List<AcronymDef>> getDefinitionsOf(@Query("sf") String sf);
	
}
	\end{lstlisting}
	\noindent
	Retrofit basiert auf OkHttp, diese Library ist also automatisch vorhanden.
	Man kann verwendete OkHttpClient-Instanzen auch konfigurieren, bevor man sie auf die Retrofit-Factory setzt.\\
	Gradle Dependencies:\\
	\texttt{com.squareup.retrofit2:retrofit:2.5.0}\\
	\texttt{com.squareup.retrofit2:converter-gson:2.5.0} (JSON Mapper)
	
	\newpage
	
	\begin{figure}[!htb]
		\centering
		\includegraphics[width=.95\textwidth]{img/retrofit_config.png}
		\caption{Konfiguration und Aufruf von Retrofit}
		\label{fig:retrofit_config}
	\end{figure}

	\begin{figure}[!htb]
		\centering
		\includegraphics[width=.95\textwidth]{img/retrofit_demo.png}
		\caption{Demo: JSON-Service mit Retrofit konsumieren}
		\label{fig:retrofit_demo}
	\end{figure}
		
\newpage
\section{Android 5 - Services \& Broadcast Receiver}

	\subsection{Service-Komponente}
	
	\begin{itemize}
		\item Services wurden ursprünglich zur Kapselung und Erledigung von Hintergrundaufgaben eingeführt. \\
		Seit API 26 (Android 8 Oreo) stark eingeschränkt $\rightarrow$ System Performance Issues (zuviele Services in zu vielen Apps)
		
		\item \textbf{Heute}: Nur als Spezialfall "Foreground Service" und zum Export der App-Logik (Sicherheitsrisiko) empfohlen (Bsp. Musikplayer)
		
		\item Für frühere Anwendungsbereiche: \textbf{WorkManager} empfohlen (Bsp. Location Update, Background Sync etc.)
	\end{itemize}

		\subsubsection{Service-Konzept}
	
		\begin{itemize}
			
			\item Was \textbf{kann} ein Service?
			
			\begin{itemize}
				
				\item Dem System mitteilen, dass eine Arbeit im Hintergrund ausgeführt werden soll \\
				\texttt{startService()} - Auftrag für Service erteilen
				
				\item Gewisse Funktionalität (API) exportieren und anderen Apps anbieten \\
				\texttt{bindService()} - öffnet stehende Verbindung für Kommunikation mit Services
				
			\end{itemize}
			
			\item Was kann ein Service \textbf{nicht}?
			
			\begin{itemize}
				
				\item Kein seperater Worker-Thread (per se)
				
				\item Kein eigener Prozess (nur wenn als solcher definiert)
				
			\end{itemize}
		
			\item Ein Service \textbf{kann und sollte} einen eigenen Thread für langandauernde Operationen starten
			
		\end{itemize}
	
		\subsubsection{Lebensarten \& Lifycycle-Methods eines Service}

		\begin{itemize}
			
			\item \textbf{Ungebundener Service:} Service verleibt im Zustand RUNNING bis er explizit beendet wird \\
			Aufruf durch \texttt{startForegroundService(...)}
			\begin{itemize}
				
				\item \texttt{onCreate()}: Bei Erzeugung
				\item \texttt{onStartCommand()}: Auftragsbehandlung
				\item \texttt{onDestroy()}: Bei Beendung (durch Service/App/System)
				
			\end{itemize}
			
			\item \textbf{Gebundener Service:} Service verbleibt nur so lange im Zustand RUNNING wie Bindings existieren\\
			Aufruf durch \texttt{bindService()}
			\begin{itemize}
				
				\item \texttt{onCreate()}: Bei Erzeugung
				\item \texttt{onBind()}: wenn Komponente Verbindung herstellt
				\item \texttt{onUnbind()}: wenn Komponente Verbindung beendet
				\item \texttt{onDestroy()}: Bei Beendung (durch Service/App/System)
				
			\end{itemize}
			
		\end{itemize}
		\vspace{1em}
		Auf der nachfolgenden Seite sind die Lifecycle-Methoden eines Service übersichtlich dargestellt.
		
		\newpage

		\begin{figure}[!htb]
			\centering
			\includegraphics[width=.7\textwidth]{img/services_lifecycle.png}
			\caption{Lifecycle-Methoden von Services\\
			\textit{Ab API 26 ist startService() $\rightarrow$ startForegroundService()}}
			\label{fig:retrofit_demo}
		\end{figure}
	
		\newpage
	
		\subsubsection{Demo-Code eines Foreground-MusicPlayer-Service}
		
		\begin{itemize}
			
			\item Hintergrundarbeit, die für den Benutzer wahrnehmbar ist
			\item Zeigt eine Notification, während der Service läuft
			\item Benötigt Permission: \texttt{FOREGROUND\_SERVICE}
			\item Interaktion / Steuerung des Service oft über Binding
			
		\end{itemize}
	
		\begin{figure}[!htb]
			\centering
			\includegraphics[width=.8\textwidth]{img/service_demo01.png}
			\caption{Demo eines Foreground Service - 01}
			\label{fig:service_demo_01}
		\end{figure}
	
		\begin{figure}[!htb]
			\centering
			\includegraphics[width=.8\textwidth]{img/service_demo02.png}
			\caption{Demo eines Foreground Service - 02}
			\label{fig:service_demo_02}
		\end{figure}
		
		\newpage
		\noindent
		\textbf{Service starten:}\\
		\textit{(Im Intent können auch Parameter mitgegeben werden!)}
		\begin{lstlisting}
public void startPlayerService(View v) { 
	startService(new Intent(this, DemoMusicPlayerService.class)); 
}
		\end{lstlisting}
		\textbf{Service stoppen:}
		\begin{lstlisting}
public void stopPlayerService(View v) { 
	stopService(new Intent(this, DemoMusicPlayerService.class)); 
}
		\end{lstlisting}
		
		\subsubsection{Allgemeines Muster \& "Stickyness" von Services}
		
		\begin{itemize}
			
			\item Ein Service wird beim App-Start oder beim Start eines Events als Foreground-Service gestartet und bleibt \textbf{alive}. 
			
			\item Der Service startet einen \textit{Worker-Thread} oder \textit{Thread-Pool}, der aktiv bleibt (zu Beginn ggf. idle).
			
			\item Mittels \texttt{bindService()} kann synchron kommuniziert werden.
			
			\item \textbf{Was soll mit einem Service passieren, wenn das System den App-Prozess zerstört und diesen später wiederherstellt?}
			\begin{itemize}
				
				\item \texttt{onStartCommand()} retourniert die gewünschte Verhaltensweise, dies ist für gebundene Service jedoch nicht wichtig
				
				Mögliche Rückgabewerte von \texttt{onStartCommand()} wären:
				\begin{itemize}
					
					\item \textbf{START\_STICKY}:\\
					Service nach Wiederherstellung automatisch wieder starten - \texttt{onStartCommand()} wird erneut aufgerufen, aber ohne Intent (Service sollte Zustand persistieren und wieder laden z.Bsp. Queue bei Musikplayer)
										
					\item \textbf{START\_NOT\_STICKY}:\\
					Service nicht automatisch neu starten nach Wiederherstellung
					
					\item \textbf{START\_REDELIVER\_INTENT}:\\
					Wie bei \textit{START\_STICKY}, der ursprüngliche Intent wird jedoch nochmals ausgeliefert, damit parametrisierte Reinitialisierung möglich ist.
					
				\end{itemize}
			\end{itemize}		
		\end{itemize}
	
		\subsubsection{Gebundene Services}
		
		\begin{itemize}
			
			\item Service kann gebunden werden mittels\\
					\texttt{bindService(intent, connection, flag)}
					
			\item Client kommuniziert mit Service über \texttt{ServiceConnection},
			damit kann die Funktionalität einer App exportiert werden
			(insbesondere mit einem "Remote Service")
			
			\item Bindung lösen mit \texttt{unbindService(connection)}
			
		\end{itemize}
	
		\begin{figure}[!htb]
			\centering
			\includegraphics[width=.8\textwidth]{img/binded_service.png}
			\caption{Verbindung bei einem gebundenen Service}
			\label{fig:bound_service}
		\end{figure}
	
		\newpage
	
		\paragraph{Involvierte Klassen bei gebundenem Service}
		
		\textit{(Die Namen der Klassen leiten sich von der dazugehörigen MOBPRO-Übung der SW05 ab und dienen bloss der Veranschaulichung der Klassen)}\\
		
		\begin{itemize}
			
			\item \textbf{Service Interface} (MusicPlayerApi)\\
			Definiert die API eines Service
			
			\item \textbf{Binder} (MusicPlayer.MusicPlayerApiBinder)\\
			Implementiert das Service Interface und wird dem Client bei einer erfolgreichen Verbindung übergeben (Service-Stub / -Handle)
			
			\item \textbf{Service Connection} (MusicPlayerConnection)\\
			Definiert die Callbacks für eine erfolgreiche/verlorene Verbindung und enthält Binder-Objekte (= API) bei einer erfolgreichen Verbindung
			
			\item \textbf{Service} (MusicPlayerService)\\
			Implementiert \texttt{onBind(intent)} und gibt ein Binder-Objekt zurück
			
			\item \textbf{Client} (MainActivity)\\
			Ruft \texttt{bindService(intent, connection}\textit{(Callback-Handler)}\texttt{, flag)} \\
			(resp. \texttt{unbindService(connection)}) auf
			
			
		\end{itemize}
	
		\begin{figure}[!htb]
			\centering
			\includegraphics[width=\textwidth]{img/binding_service_ablauf.png}
			\caption{Ablauf Service-Bindung \& Service API-Abruf}
			\label{fig:bound_service_ablauf}
		\end{figure}
	
		\newpage
		
		\subsubsection{Fortführung Demo Service}
		
		\begin{figure}[!htb]
			\centering
			\includegraphics[width=.8\textwidth]{img/service_demo04.png}
			\caption{Demo eines Foreground Service - 04}
			\label{fig:service_demo_04}
		\end{figure}
	
		\begin{figure}[!htb]
			\centering
			\includegraphics[width=.8\textwidth]{img/service_demo05.png}
			\caption{Demo eines Foreground Service - 05}
			\label{fig:service_demo_05}
		\end{figure}
	
	\newpage
	
	\subsection{Broadcast Receiver}
	
	\begin{itemize}
		\item Broadcasts sind Nachrichten, es gibt einen app-internen "Message Bus"
		
		\item Alle Komponenten einer App können Broadcasts versenden und sich für den Empfang registrieren
		
		\item Das System selbst versendet ebenfalls Nachrichten bei gewissen Events (App installiert, Timer etc.)
		
		\item Aus Performancegründen stark eingeschränkt seit API 26, es werden nur noch sehr wenige Events global verteilt 
	\end{itemize}

		\subsubsection{Broadcasts versenden}
		
		\begin{itemize}
			
			\item Broadcasts werden als Intents versendet
			\begin{itemize}
				\item \textbf{Implizit} in der App via \\
				 \texttt{LocalBroadcastManager.getInstance(this).sendBroadcast(intent)}
				 
				 \item \textbf{Explizit} an andere Apps über \\
				 \texttt{Context::sendBroadcast(intent)}
			\end{itemize}
			
			\item Empfang von Broadcasts: Receiver (Empfänger)
			\begin{itemize}
				\item Empfänger werden dynamisch im Code registriert:\\
				\texttt{registerReceiver(receiver, filter)}
				
				\item Können statisch im Manifest deklariert werden:\\
				Tag \texttt{<receiver ...>} $\leftarrow$ nur explizite Broadcasts, auch wenn App nicht gestartet
			\end{itemize}
			
		\end{itemize}

		\subsubsection{Globaler Broadcast Receiver}
		
		\begin{itemize}
			
			\item Ein Broadcast Receiver ist immer nur so lange aktiv, wie die Bearbeitung der empfangenen Nachricht braucht.\\
			(\textbf{DON'T:} keine AsyncTasks, keinen Service binden, keine Dialoge anzeigen!\\
			\textbf{DO:} Activity starten, Service starten, Notification senden)
			
			\begin{itemize}
				\item Ansonsten wird er inaktiv und vom System gelöscht
				\item Erneute Erzeugung nur auf Abruf
				\item \textbf{Nur} für den Empfang von expliziten Broadcasts geeignet
			\end{itemize}
			
			\item Broadcast Receiver hat kein UI:\\
			Notifications für Kommunikation mit User benutzen oder einen Service starten für Hintergrundaufgaben
			
		\end{itemize}
	
		\begin{figure}[!htb]
			\centering
			\includegraphics[width=.8\textwidth]{img/br_manifest.jpg}
			\caption{Eintragen eines Receivers im Manifest}
			\label{fig:br_manifest}
		\end{figure}
		\noindent
		Dedizierter Broadcast Receiver Klasse erstellen:
		
		\begin{lstlisting}
public class BootCompletedReceiver extends BroadcastReceiver {
	@Override
	public void onReceive(Context context, Intent intent) {
			// do something, when boot has completed, e.g. start a service...
	}
}
		\end{lstlisting}
		\noindent
		Expliziten Broadcast an eine andere App versenden:
		
		\begin{lstlisting}
Intent broadcastIntent = new Intent("ACTION_MY_BROADCAST");
broadcastIntent.setPackage("ch.hslu.mobpro.other"); // Empfaenger ID
sendBroadcast(broadcastIntent)
		\end{lstlisting}
		
		\newpage
		
		\subsubsection{Lokale Broadcasts - "App Message Bus"}
		
		Broadcast Receiver erzeugen \& registrieren im Code:
		
		\begin{lstlisting}
downloadCompleteListener = new BroadcastReceiver() {
	@Override
	public void onReceive(Context context, Intent intent) {
		Toast.makeText(this, "Got it!", LENGTH_SHORT).show();
	}
};

IntentFilter filter = new IntentFilter("mobpro.DOWNLOAD_COMPLETE");
LocalBroadcastManager.getInstance(this)
	.registerReceiver(downloadCompleteListener, filter);
		\end{lstlisting}
		\noindent
		Nachricht versenden (Emitter): 
		
		\begin{lstlisting}
Intent downloadComplete = new Intent("mobpro.DOWNLOAD_COMPLETE");
downloadComplete.putExtra("file", "Terminator2.mp4"); LocalBroadcastManager.getInstance(this).sendBroadcast(downloadComplete);
		\end{lstlisting}
		\noindent
		Broadcast Receiver deregistrieren (wenn er nicht mehr benötigt wird): 
		
		\begin{lstlisting}
LocalBroadcastManager.getInstance(this) 
	.deregisterReceiver(downloadCompleteListener);
		\end{lstlisting}
	
	\newpage
	
	\subsection{Work Manager \& Broadcasts}
	
	\begin{itemize}
		\item \textbf{WorkManager für Hintergrundtasks}\\
				Repetitive oder einmalige Background-Tasks, die aufschiebbar sind, sollten via WorkManager erledigt werden.
				
		\item Wie findet die App heraus, wenn der Task abgeschlossen ist?\\
		$\rightarrow$ z.Bsp. mittels einem lokalen Broadcast (wird nur verarbeitet, wenn die App noch läuft bzw. solange der Receiver noch registriert ist, bspw. um eine Liste von heruntergeladenen Dateien zu aktualisieren
		
		\item Beispielablauf mit WorkManager:
		\begin{itemize}
			\item Es ist ein lang andauernder Task vorhanden, bspw. Positionen von Objekten erkennen
			
			\item Dieser Worker-Task wird an den WorkManager übergeben
			
			\item Sobald die Positionen bestimmt wurden, werden diese mittels Broadcast zurückgemeldet
			
			\item Der Broadcast-Receiver hört auf diese Action und zeigt die Resultate in einem Toast
		\end{itemize}
	
	\item Gradle Dependency:\\
			\texttt{androidx.work:work-runtime:2.0.0-rc01}
			
	\item Nachfolgendes Beispiel funktioniert auch mit Threads 
	
	\item Worker kann auch als wiederkehrender Task (auch mit anfänglichem Delay) registriert werden
	
	\item Arbeiten können auch in Graphen mit Abhängigkeiten definiert werden
	
	\end{itemize}

	\begin{figure}[!htb]
		\centering
		\includegraphics[width=.8\textwidth]{img/workmanager_beispiel.png}
		\caption{Beispielcode für einen WorkManager}
		\label{fig:workmanager_beispiel}
	\end{figure}
		
\newpage
\section{Android 6 - Intents, Fragments, App-Widgets}

	\subsection{Intent Filters}
	
	\begin{itemize}
		
		\item Bei \textbf{impliziten Intents}:\\
				Steht der Empfänger nicht im Vornherein fest, es gibt nur eine Nachricht mit einem "Anforderungsprofil" für den Empfänger
				
		\item Das System eruiert mögliche Empfänger (Intent Resolution)\\
				Drei mögliche Fälle:
		\begin{itemize}
			
			\item Genau ein Empfänger gefunden $\rightarrow$ direkte Zustellung
			
			\item Mehrere Empfänger $\rightarrow$ Auswahl durch User per Dialog
			
			\item Kein Empfänger $\rightarrow$ \texttt{ActivityNotFoundException}
						
		\end{itemize}
		
		\item Potentielle Intent-Empfänger deklarieren Intent-Filter im Manifest (oder deklarieren es in einem Broadcast Receiver)
		
		\item Das System vergleicht implizite Intents mit deklarierten Filtern und liefert die passenden Komponenten zurück\\
		(Der Benutzer kann wählen, falls mehrere Filter passen und kein Favorit registriert ist)
				
	\end{itemize}

	\begin{figure}[!htb]
		\centering
		\includegraphics[width=.6\textwidth]{img/android6/intentfilter_manifest.png}
		\caption{Intent-Filter im Manifest deklarieren}
		\label{fig:intentfilter_manifest}
	\end{figure}

	\textit{Beispiel für eigene Intent-Action in den Vorlesungsfolien}

		\subsubsection{Implizite Intents: Daten}
		
		Implizite Intents können (optional) folgende Daten enthalten:
		
		\begin{itemize}
			
			\item \textbf{Action}\\
					Typ der Aktion, welche ausgeführt werden soll\\
					Bsp. \texttt{ACTION\_VIEW, ACTION\_EDIT, CUSTOM\_ACTION}...
					
			\item \textbf{Category}\\
			Kategorie der Komponente, welche diesen Intent ausführen soll\\
			Bsp. \texttt{DEFAULT, LAUNCHER, BROWSABLE}...
			
			\item \textbf{Data}\\
			Beschreibung der Daten, mit welchen gearbeitet werden soll\\
			(URI und Mime Type)
			
			\item \textbf{Extras}\\
			Schlüssel / Wert-Paare für Zusatzinformationen
			
		\end{itemize}
		
		\newpage
		
		\subsubsection{Implizite Intents: Auflösung von Intents}
		
		Das Android-System löst implizite Intents auf, indem die am besten zum Intent passende Komponente ausgewählt wird ("Best Match"). Diese wird festgelegt durch Vergleich von:
			
		\begin{itemize}
			
			\item \textbf{Action}: Action des Intents muss im Filter sein
			
			\item \textbf{Category}: Jede Kategorie des Intents muss im Filter sein
			
			\item \textbf{Data (URI \& Mime Type}: Alles im Intent unter \textit{"Data"} Aufgelistete muss zum Filter passen
			
		\end{itemize}
		
		\begin{figure}[!htb]
			\centering
			\includegraphics[width=.65\textwidth]{img/android6/intents_implizit.jpg}
			\caption{Übertragen von Impliziten Intents zw. Applikationen (Google Doku)}
			\label{fig:implicit_intents}
		\end{figure}
		\noindent
		Infos zu aktuell installierten Packages im Android-System können mit folgender Klasse abgefragt werden:\\
		\textbf{\texttt{PackageManager}}\\
		\quad \texttt{.query...()}: liefert alle passenden Intent-Auflösungen zurück\\
		\quad \texttt{.resolve...()}: liefert die best-passenden Intent-Auflösungen zurück\\
		\\
		Beispiel, um Activities für einen Intent abzufragen:
		\begin{lstlisting}
	final List<ResolveInfo> resolveList = getPackageManager()
			.queryIntentActivites(getHsluBrowserIntent(),
					PackageManager.MATCH_DEFAULT_ONLY);
		\end{lstlisting}
		Aus dieser Liste für alle \texttt{ResolveInfo} das Feld \texttt{activityInfo.name} ausgeben, um alle verfügbaren Activities anzuzeigen.
	
	\newpage
	
	\subsection{Fragments}
	
	\begin{itemize}
		\item \textbf{Fragments}:\\
				Modularer (UI-)Teil innerhalb einer Activity
		\item \textbf{Eigenschaften}:\\
				Eigener Lebenszyklus, eigener Zustand, kann einer laufenden Activity hinzugefügt / entfernt werden (Transaktion)
		\item Ist quasi eine Art "Sub-Activity", welche in verschiedenen Activities wiederverwendet werden kann
	\end{itemize}

	\begin{figure}[!htb]
		\centering
		\includegraphics[width=.3\textwidth]{img/android6/fragment_lifecycle.jpg}
		\caption{Lebenszyklus eines Fragments}
		\label{fig:fragment_lifecycle}
	\end{figure}

	\begin{itemize}
		
		\item Fragments werden an Activities angehängt (attached)
		
		\item Fragments können im \texttt{layouts.xml} deklariert oder programmatisch erzeugt werden
		
		\item \textit{FragmentManager} verwaltet die Fragments innerhalb einer Activity:\\
				\texttt{Activity.getFragmentManager()}, mit Support Package:\\
				\texttt{FragmentActivity.getFragmentManager()}
		
	\end{itemize}

		\newpage
		
		\subsubsection{Fragments hinzufügen}
	
		Mit dem \texttt{LayoutInflater} kann ein Fragment-UI aus einem \texttt{layout.xml} erzeugt werden. Die von Fragment abgeleitete Klasse implementiert die zugehörige Logik.
		
		\begin{lstlisting}
public static class ExampleFragment extends Fragment {
	@Override
	public View onCreateView(LayoutInflater inflater, ViewGroup container, 
		Bundle savedInstanceState) {
			return inflater.inflate(R.layout.example_fragment, container, false);
	}
}
		\end{lstlisting}
		\noindent
		In einer Activity können Fragments programmatisch (ggf. im \texttt{onCreate()}) hinzugefügt werden.
		Dies passiert mittels \texttt{FragmentTransactions}, welche vom \texttt{FragmentManager} verwaltet werden.\\
		\textit{(fragmentContainer ist in diesem Beispiel ein FrameLayout im layout.xml der Activity)}
		
		\begin{lstlisting}
getFragmentManager()
		.beginTransaction()
		.add(R.id.fragmentContainer, FavColorFragment.newInstance())
		.commit();
		\end{lstlisting}
		\noindent
		Fragments können auch direkt im layout.xml einer Activity deklariert werden:
		
		\begin{figure}[!htb]
			\centering
			\includegraphics[width=.7\textwidth]{img/android6/fragment_xml.png}
			\caption{Deklaration eines Fragments im layout.xml}
			\label{fig:fragment_xml}
		\end{figure}
		\noindent
		\textbf{Zum Vorlesungsbeispiel}: Einstellungen des Fragments werden in SharedPreferences gespeichert, wobei die Prefs in \texttt{onResume()} wieder geladen werden, damit diese sich bei einem Wechsel synchronisieren.
	
	\subsection{App-Widgets}
	
	\begin{itemize}
		
		\item App-Widgets sind quasi "Mini-Apps" auf dem Homescreen eines Androidgeräts, zeigen dort wichtige Daten und Funktionalitäten einer App an
		
		\item Es gibt verschiedene Typen: Information / Collection / Control / Hybrid-Widgets
		
		\item \textbf{How-To für Widgets}
		\begin{itemize}
			
			\item Widget im Android-Manifest als solches deklarieren\\
					$\rightarrow$ Widgets sind spezielle \texttt{BroadcastReceiver}, werden als Receiver-Komponenten deklariert
					
			\item \texttt{AppWidgetProviderInfo} muss angegeben werden:
			\begin{itemize}
				\item \textit{Deklarativ}: widget\_provider\_info.xml
				\item \textit{Programmatisch}: Klasse \texttt{AppWidgetProviderInfo}
			\end{itemize}
			
			\item UI für das Widget in einer \texttt{widget.xml} Datei
			
			\item Eigene von \texttt{AppWidgetProvider} abgeleitete Klasse, behandelt Aktualisierungslogik etc.\\
			
		\end{itemize}	
	
		\item \textbf{Codebeispiele und wie ein Force Update für ein Widget programmiert wird, ist auf den Folien zu "MobPro-Android\_6" auf Seiten 29-34 ersichtlich.}
		
	\end{itemize}
	
	\newpage
	
	\subsection{App-Design}
	
	Unter \href{https://developer.android.com/design/}{\textit{Design für Android}} und \href{https://material.io/design/}{\textit{Material Design}} können die ständig aktualisierten Design-Vorgaben eingesehen werden. 
	Für wichtigste Interaktionen verwenden Apps auf Android oft die \href{https://developer.android.com/training/appbar/index.html}{App-Bar}.
	Das \href{https://romannurik.github.io/AndroidAssetStudio/index.html}{Android Asset Studio} bietet zudem eine Auswahl an Tools, um einfach Assets wie Launcher- / Notification- / App-Bar-Icons und vieles mehr zu erstellen.
	
	\subsection{Usability \& Prototyping}
	
	\begin{itemize}
		
		\item Prototypen helfen beim Design (iterativer Prozess)
		
		\item Kommunikation mit Kunden, der Benutzer sieht etwas\\
		Verständnis für EntwicklerIn für die App/Anwendungsdomäne
		
		\item Stufen im Designmodell:\\
		($\rightarrow$ Analyze $\leftrightarrow$ Design $\leftrightarrow$ Prototype $\rightarrow$ Evaluate $\rightarrow$) $\rightarrow$ Final Product
		
		\item Prototypen in Software oder auf Papier, durcharbeiten für besseren Überblick der App-Funktionalität und -Navigation (Wireframes / Mocks / Storyboard ...)
		
		\item Android-Doku zu Wireframes zur Vertiefung anschauen!
		
	\end{itemize}

	\subsection{Android Jetpack \& Support Library (AppCompat)}
	
	\begin{quote}
		"\textbf{When developing apps that support multiple API versions}, you may want a standard way to \textbf{provide newer features on earlier versions of Android} or gracefully fall back to equivalent functionality. Rather than building code to handle earlier versions of the platform, you can leverage these libraries to provide that compatibility layer. In addition, the Support Libraries provide additional convenience classes and features not available in the standard Framework API for easier development and support across more devices. Originally a single binary library for apps, the Android Support Library has evolved into a suite of libraries for app development. Many of these libraries are now a strongly recommended, if not essential, part of app development."
	\end{quote}

		\paragraph{Beispiel: Action Bar (App Bar)}
		
		Die Action Bar gibt es erst seit API 11, ist dank der Support Library aber schon ab API-Level 7 (Android 2.1) verfügbar (Warnhinweis dazu in der Support Library Doku).
		
		\paragraph{Android Jetpack}
		
		Ist verfügbar seit API 28, enthält die AppCompat-Library, man muss jedoch seine App zu AndroidX migrieren (wird wohl erst ab Android 10 standardmässig integriert sein), Infos dazu auf entsprechender Android-Doku Webseite.\\
		Beispiele für Hilfslibriaries: \textit{Databinding, LiveData, ViewModel, Room, Download Manager, Notifications, Sharing, Animation \& Transitions, Layout, ...}
	
	\newpage
	
	\subsection{Publizieren von Android-Apps}
	
	Primäre Quellen für App-Veröffentlichung unter \textit{Nützliche Links}\\
	Jeder kann Apps im Google Play Store veröffentlichen, Voraussetzungen dafür sind:
	\begin{itemize}
		\item gratis Google Account
		\item Google Play Publisher Account (25\$, einmalig), neuen Google Acc dafür erstellen
		\item Google Payments Merchant Account (nur für kostenpflichtige Apps, 30\% an Google, 70\% an EntwicklerIn)\\
	\end{itemize}
	\noindent
	Grob-Vorgehen bei der Veröffentlichung:
	
	\begin{enumerate}
		\item \textbf{Code Cleanup, Release vorbereiten}\\
				Verionierung, Internationalisierung, Logging/Debugging entfernen etc.
				
		\item \textbf{Release erstellen \& signieren} (Dev. Identity o.ä.)
		
		\item \textbf{Werbematerial vorbereiten}\\
				Screenshots, Hi-Res-Icons, Promo, Website, Video, Google Ads etc.
				
		\item \textbf{Distributionseinstellungen setzen}\\
				Content Rating, Verfügbarkeit, Grösse, Bezahlungsart (gratis, paid, in-app)
				
		\item \textbf{Release Upload \& Veröffentlichung}\\
		
		\item Link: \href{https://developer.android.com/distribute/best-practices/launch/launch-checklist.html}
		{\textit{Offizielle Google Launch Checklist}}\\
		
	\end{enumerate}
	\noindent
	Veröffentlicht wird eine APK-Datei (kurz: signiertes Archiv mit einer App drin)\\
	\textbf{ProGuard} ist ein Optmierer, welchen Code schwer lesbar und kleiner macht und ist per Default im Release-Modus aktiviert. Wird in Zukunft wahrscheinlich durch Google's eigenen R8 Optimierer ersetzt.
	
		\subsubsection{Signieren von Apps}
		
		\begin{itemize}
			
			\item Play-Store Apps müssen signiert sein\\ 
			(Nicht mit (default) Developer Key aus der IDE)
			
			\item Self-Signing ist erlaubt (kein autorisiertes Zertifikat notwendig)
			
			\item Benötigt wird ein Keystore mit Private Keys
			
			\item Android Studio Wizard hilft beim Erstellen eines Keystores mit privatem Schlüssel ( = Zertifikat)
			
			\item Updates müssen mit dem \textbf{gleichen} Zertifikat signiert sein (produktiven Schlüssel nicht verlieren!)
			\begin{itemize}
				\item Wenn ja: problemloses Update möglich
				\item Wenn nein: dann muss App mit neuem Package-Namen als komplett neue App installiert werden
			\end{itemize}
			
		\end{itemize}
	
		\subsubsection{Andere Distributionsmöglichkeiten}
		
		Neben Google Play Store können Apps in anderen Stores oder über eine eigene Website oder andere Quelle verteilt werden. Das Gerät, welches die APK installieren möchte, muss dabei aber die Installation aus unbekannten Quellen in dein Einstellungen erlauben.
		
		\subsubsection{Test before Release}
		
		Apps manuell und automatisch testen. 
		Manuell auf mindestens einem Gerät, im besten Falle auf mehreren Geräten mit unterschiedlichen Bildschirmauflösungen, Android-Versionen und verschiedenen Herstellern.
		Grosser Aufwand, dass eine App auf möglichst vielen Geräten lauffähig ist!
	
\newpage
\section{Android 7 - Hybrid WebApp}

	\subsection{Mobile Web-Anwendungen}
	
	

\section{Android 8 - Entwicklungsansätze}

	% TODO
	
	
	
\newpage
	
\section{Nützliche Links}

	\subsection{Android 1 - Grundlagen}
	
	\begin{itemize}
		\item \textbf{Referenz-Liste aller Android-Versionen}\\
		\href{https://de.wikipedia.org/wiki/Liste_von_Android-Versionen}
		{https://de.wikipedia.org/wiki/Liste\_von\_Android-Versionen}
		
		\item \textbf{Android Developers | Startseite}\\
		\href{https://developer.android.com/}
		{https://developer.android.com/}
		
		\item \textbf{Android Developers | Developer Guide}\\
		\href{https://developer.android.com/guide}
		{https://developer.android.com/guide}
		
		\item \textbf{Android Developers | Package Index (APIs)}\\
		\href{https://developer.android.com/reference
		}{https://developer.android.com/reference}
		
		\item \textbf{Android Developers | Android Platform Architecture}\\
		\href{https://developer.android.com/guide/platform/}
		{https://developer.android.com/guide/platform/}
	\end{itemize}
	
	\subsection{Android 2 - Benutzerschnittstellen}
	
	\begin{itemize}
		\item \textbf{Android Developers | ViewGroup.LayoutParams}\\
		\href{https://developer.android.com/reference/android/view/ViewGroup.LayoutParams.html}
		{https://developer.android.com/reference/android/view/ViewGroup.LayoutParams.html}
		
		\item \textbf{Android Developers | Unterschiedliche Screen Sizes}\\
		\href{https://developer.android.com/training/multiscreen/screensizes.html}
		{https://developer.android.com/training/multiscreen/screensizes.html}
		
		\item \textbf{Android Developers | Debug mit Layout Inspector}\\
		\href{https://developer.android.com/studio/debug/layout-inspector.html}
		{https://developer.android.com/studio/debug/layout-inspector.html}
		
		\item \textbf{Responsive UI mit ConstraintLayout}\\
		\href{https://developer.android.com/training/constraint-layout/index.html}
		{https://developer.android.com/training/constraint-layout/index.html}
		
		\item \textbf{Medium | Einführung in Android's ConstraintLayout}\\
		\href{https://medium.com/exploring-android/exploring-the-new-android-constraintlayout-eed37fe8d8f1}
		{https://medium.com/exploring-android/exploring-the-new-android-constraintlayout-eed37fe8d8f1}
		
		\item \textbf{Android Developers | LinearLayout}\\
		\href{https://developer.android.com/reference/android/widget/LinearLayout.LayoutParams.html}
		{https://developer.android.com/reference/android/widget/LinearLayout.LayoutParams.html}
		
		\item \textbf{Android Developers | Screen Compatibility}\\
		\href{https://developer.android.com/guide/practices/screens_support.html#qualifiers}
		{https://developer.android.com/guide/practices/screens\_support.html\#qualifiers}
		
		\item \textbf{Wikipedia | ISO Language Codes}\\
		\href{https://en.wikipedia.org/wiki/List_of_ISO_639-1_codes}
		{https://en.wikipedia.org/wiki/List\_of\_ISO\_639-1\_codes}
		
		\item \textbf{Wikipedia | ISO Country Codes}\\
		\href{https://en.wikipedia.org/wiki/ISO_3166-1}
		{https://en.wikipedia.org/wiki/ISO\_3166-1}
		
		\item \textbf{Android Developers | Best-Matching Resource}\\
		\href{https://developer.android.com/guide/topics/resources/providing-resources.html#BestMatch}
		{https://developer.android.com/guide/topics/resources/providing-resources.html\#BestMatch}
		
		\item \textbf{Android Developers | android.widget (View-Klassen APIs Summary)}\\
		\href{https://developer.android.com/reference/android/widget/package-summary.html}
		{https://developer.android.com/reference/android/widget/package-summary.html}
		
		\item \textbf{Android Developers | Data Binding Library}\\
		\href{https://developer.android.com/topic/libraries/data-binding}
		{https://developer.android.com/topic/libraries/data-binding}
		
		\item \textbf{Android Developers | eigenes CustomToastView Layout}\\
		\href{https://developer.android.com/guide/topics/ui/notifiers/toasts.html#CustomToastView}
		{https://developer.android.com/guide/topics/ui/notifiers/toasts.html\#CustomToastView}
		
		\item \textbf{Android Developers | DialogFragment}\\
		\href{https://developer.android.com/guide/topics/ui/dialogs.html#DialogFragment}
		{https://developer.android.com/guide/topics/ui/dialogs.html\#DialogFragment}
		
		
	\end{itemize}
	
	\subsection{Android 3 - Persistenz}
	
	\begin{itemize}
		\item \textbf{Android Developers | Permissions (Übersicht)}\\
		\href{https://developer.android.com/guide/topics/permissions/overview}
		{https://developer.android.com/guide/topics/permissions/overview}
		
		\item \textbf{Github | PermissionsDispatcher}\\
		\href{https://github.com/permissions-dispatcher/PermissionsDispatcher}
		{https://github.com/permissions-dispatcher/PermissionsDispatcher}
		
		\item \textbf{Android Developers | SQLite \textit{(nicht empfohlen)}}\\
		\href{https://developer.android.com/training/data-storage/sqlite.html}
		{https://developer.android.com/training/data-storage/sqlite.html}
		
		\item \textbf{Android Developers | Room}\\
		\href{https://developer.android.com/training/data-storage/room/index.html}
		{https://developer.android.com/training/data-storage/room/index.html}
		
		\item \textbf{Android Developers | RecyclerView}\\
		\href{https://developer.android.com/guide/topics/ui/layout/recyclerview}
		{https://developer.android.com/guide/topics/ui/layout/recyclerview}
		
		\item \textbf{CodeLabs | Android Room with a View(Model)}\\
		\href{https://codelabs.developers.google.com/codelabs/android-room-with-a-view/#0}
		{https://codelabs.developers.google.com/codelabs/android-room-with-a-view/\#0}
		
		\item \textbf{Android Developers | Room - Queries in Klassen kapseln}\\
		\href{https://developer.android.com/training/data-storage/room/creating-views}
		{https://developer.android.com/training/data-storage/room/creating-views}
		
		\item \textbf{Android Developers | Room - Observable Queries mit LiveData}\\
		\href{https://developer.android.com/training/data-storage/room/accessing-data#query-observable}
		{https://developer.android.com/training/data-storage/room/accessing-data\#query-observable}
		
		\item \textbf{Android Developers | Room - Datenbank migrieren (bspw. bei App-Updates}\\
		\href{https://developer.android.com/training/data-storage/room/migrating-db-versions}
		{https://developer.android.com/training/data-storage/room/migrating-db-versions}
		
		\item \textbf{Android Developers | Room - Datenbank testen}\\
		\href{https://developer.android.com/training/data-storage/room/testing-db}
		{https://developer.android.com/training/data-storage/room/testing-db}
		
		\item \textbf{Android Developers | Room - TypeConverter: Objekt-Referenzen in DB}\\
		\href{https://developer.android.com/training/data-storage/room/referencing-data}
		{https://developer.android.com/training/data-storage/room/referencing-data}
		
		\item \textbf{Android Developers | Calendar Provider}\\
		\href{https://developer.android.com/guide/topics/providers/calendar-provider.html}
		{https://developer.android.com/guide/topics/providers/calendar-provider.html}
		
		\item \textbf{Android Developers | Contacts Provider}\\
		\href{https://developer.android.com/guide/topics/providers/contacts-provider.html}
		{https://developer.android.com/guide/topics/providers/contacts-provider.html}
		
		
	\end{itemize}
	
	\subsection{Android 4 - Kommunikation \& Nebenläufigkeit}
	
	\begin{itemize}
		
		\item \textbf{Android Developers | ANR - Keep your App responsive}\\
		\href{https://developer.android.com/training/articles/perf-anr.html}
		{https://developer.android.com/training/articles/perf-anr.html}
		
		\item \textbf{Android Developers | Background Processing}\\
		\href{https://developer.android.com/guide/background/}
		{https://developer.android.com/guide/background/}
		
		\item \textbf{Android Developers | Threading}\\
		\href{https://developer.android.com/topic/performance/threads}
		{https://developer.android.com/topic/performance/threads}
		
		\item \textbf{Android Developers | Prozesse \& Threads (Übersicht)}\\
		\href{https://developer.android.com/guide/components/processes-and-threads.html}
		{https://developer.android.com/guide/components/processes-and-threads.html}
		
		\item \textbf{Android Developers | AsyncTask}\\
		\href{https://developer.android.com/reference/android/os/AsyncTask?hl=en}
		{https://developer.android.com/reference/android/os/AsyncTask?hl=en}
		
		\item \textbf{Android Developers | Verhaltensänderungen aller Apps}\\
		\href{https://developer.android.com/about/versions/pie/android-9.0-changes-all?hl=en}
		{https://developer.android.com/about/versions/pie/android-9.0-changes-all?hl=en}
		
		\item \textbf{HTTP Definition}\\
		\href{https://tools.ietf.org/html/rfc2616}
		{https://tools.ietf.org/html/rfc2616}
		
		\item \textbf{Github | OkHttpClient}\\
		\href{http://square.github.io/okhttp/}
		{http://square.github.io/okhttp/}
		
		\item \textbf{Github | Retrofit}\\
		\href{https://square.github.io/retrofit/}
		{https://square.github.io/retrofit/}
		
		\item \textbf{JSON Informations}\\
		\href{http://json.org}
		{http://json.org}
		
		\item \textbf{JSON Formatter \& Validator}\\
		\href{https://jsonformatter.curiousconcept.com/}
		{https://jsonformatter.curiousconcept.com/}
		
		\item \textbf{Acronime REST Service Doku}\\
		\href{http://www.nactem.ac.uk/software/acromine/rest.html}
		{http://www.nactem.ac.uk/software/acromine/rest.html}
		
	\end{itemize}
	
	\subsection{Android 5 - Services, Broadcast Receiver}
	
	\begin{itemize}
		
		\item \textbf{Android Developers | Services Overview}\\
		\href{https://developer.android.com/guide/components/services.html}
		{https://developer.android.com/guide/components/services.html}
		
		\item \textbf{Android Developers | Service}\\
		\href{https://developer.android.com/reference/android/app/Service.html}
		{https://developer.android.com/reference/android/app/Service.html}		
		
		\item \textbf{Android Developers | BroadcastReceiver}\\
		\href{https://developer.android.com/reference/android/content/BroadcastReceiver}
		{https://developer.android.com/reference/android/content/BroadcastReceiver}	
		
		\item \textbf{Android Developers | WorkManager}\\
		\href{https://developer.android.com/topic/libraries/architecture/workmanager}
		{https://developer.android.com/topic/libraries/architecture/workmanager}			
		
	\end{itemize}
	
	\subsection{Android 6 - Intents, App-Widgets, Fragments etc.}
	
	\begin{itemize}
		
		\item \textbf{Android Developers | Intents \& Intent Filters}\\
		\href{https://developer.android.com/guide/components/intents-filters.html}
		{https://developer.android.com/guide/components/intents-filters.html}
		
		\item \textbf{Android Developers | PackageManager}\\
		\href{https://developer.android.com/reference/android/content/pm/PackageManager.html}
		{https://developer.android.com/reference/android/content/pm/PackageManager.html}
		
		\item \textbf{Android Developers | Fragments}\\
		\href{https://developer.android.com/guide/components/fragments.html}
		{https://developer.android.com/guide/components/fragments.html}
		
		\item \textbf{Android Developers | App Widgets Overview}\\
		\href{https://developer.android.com/guide/topics/appwidgets/overview}
		{https://developer.android.com/guide/topics/appwidgets/overview}
		
		\item \textbf{Android Developers | Build an App Widget}\\
		\href{https://developer.android.com/guide/topics/appwidgets/index.html}
		{https://developer.android.com/guide/topics/appwidgets/index.html}
		
		\item \textbf{Android Developers | Design for Android}\\
		\href{https://developer.android.com/design/}
		{https://developer.android.com/design/}
		
		\item \textbf{Material Design}\\
		\href{https://material.io/design/}
		{https://material.io/design/}
		
		\item \textbf{Android Asset Studio}\\
		\href{https://romannurik.github.io/AndroidAssetStudio/index.html}
		{https://romannurik.github.io/AndroidAssetStudio/index.html}
		
		\item \textbf{Android Developers | Add the App Bar}\\
		\href{https://developer.android.com/training/appbar/index.html}
		{https://developer.android.com/training/appbar/index.html}
		
		\item \textbf{Android Developers | Navigation (Wireframing)}\\
		\href{https://developer.android.com/guide/navigation#wireframe}
		{https://developer.android.com/guide/navigation\#wireframe}
		
		\item \textbf{Android Developers | Support Library}\\
		\href{https://developer.android.com/topic/libraries/support-library/index.html}
		{https://developer.android.com/topic/libraries/support-library/index.html}
		
		\item \textbf{Android Developers | Support Library Setup}\\
		\href{https://developer.android.com/topic/libraries/support-library/setup}
		{https://developer.android.com/topic/libraries/support-library/setup}
		
		\item \textbf{Android Developers | Android Jetpack (AndroidX}\\
		\href{https://developer.android.com/jetpack}
		{https://developer.android.com/jetpack}
		
		\item \textbf{Android Developers | Migrating to AndroidX}\\
		\href{https://developer.android.com/jetpack/androidx/migrate}
		{https://developer.android.com/jetpack/androidx/migrate}
		
		\item \textbf{Android Developers | Publish your App}\\
		\href{https://developer.android.com/studio/publish}
		{https://developer.android.com/studio/publish}
		
		\item \textbf{Android Developers | Launch your app worldwide}\\
		\href{https://developer.android.com/distribute/best-practices/launch}
		{https://developer.android.com/distribute/best-practices/launch}
		
		\item \textbf{Android Developers | Alternative distribution options}\\
		\href{https://developer.android.com/distribute/marketing-tools/alternative-distribution}
		{https://developer.android.com/distribute/marketing-tools/alternative-distribution}
		
		\item \textbf{Android Developers | Launch checklist}\\
		\href{https://developer.android.com/distribute/best-practices/launch/launch-checklist.html}
		{https://developer.android.com/distribute/best-practices/launch/launch-checklist.html}
		
		\item \textbf{Wikipedia | Android Application Package}\\
		\href{https://en.wikipedia.org/wiki/Android_application_package}
		{https://en.wikipedia.org/wiki/Android\_application\_package}
		
		\item \textbf{ProGuard}\\
		\href{https://stuff.mit.edu/afs/sipb/project/android/sdk/android-sdk-linux/tools/proguard/docs/index.html#manual/introduction.html}
		{https://stuff.mit.edu/afs/sipb/project/android/sdk/android-sdk-linux/tools\\ \quad /proguard/docs/index.html\#manual/introduction.html}
		
		\item \textbf{Guardsquare | ProGuard and R8: Comparison of Optimizers}\\
		\href{https://www.guardsquare.com/en/blog/proguard-and-r8}
		{https://www.guardsquare.com/en/blog/proguard-and-r8}
		
		\item \textbf{Text}\\
		\href{link}
		{link}
		
	\end{itemize}

\end{document}