\documentclass[a4paper]{article}

\usepackage[margin=80pt]{geometry}
\usepackage[ngerman]{babel}
\usepackage[T1]{fontenc}
\usepackage[utf8]{inputenc}

\title{\textbf{SWDE - Software Development\\
Zusammenfassung FS 2019}}
\date{\today}
\author{Maurin D. Thalmann}

\begin{document}
	\pagenumbering{gobble}
	\maketitle
	\newpage
	\pagenumbering{arabic}
	\tableofcontents
	\newpage
	
	\section{Buildautomatisation}
	
		\subsection{Sie kennen die Vorteile eines automatisierten Buildprozesses}
			\begin{itemize}
				\item Automatisierter Ablauf, keine Interaktion mehr benötigt
				\item Reproduzierbare Ergebnisse
				\item lange Builds können auch über Nacht laufen
				\item Unabhängig von Entwicklungsumgebung
			\end{itemize}
		
		\subsection{Sie können verschiedene Beispiele von Buildwerkzeugen benennen}
			\begin{description}
				\item[Make] (für C/C++ Projekte), Urvater der Build Tools, \\
				hohe Flexibilität, gewöhnungsbedürftige Syntax
				\item[Ant] Java mit XML
				\item[Maven] Java mit XML
				\item[Buildr] Ruby-Script
				\item[Gradle] Groovy Script mit DSL
				\item[Bazel] Java mit Python-like Scripts
			\end{description}
		
		\subsection{Sie beherrschen die Anwendung eines ausgewählten Buildwerkzeuges (Apache Maven)}
		Beherrschen muss man es selber, es kann entweder aus der Shell (Terminal/Konsole) verwendet werden oder aus den integrierten Funktionen in der IDE selbst.
		
		\subsection{Sie sind mit den wesentlichen Konzepten von Apache Maven vertraut}
		Deklaration des Projektes in XML, zentrales Element pro Projekt ist das \textbf{Project Object Model (POM)}, welches Metainformationen, Plugins und Dependencies definiert. Basiert auf einem globalen, binären Repository. Plugins werden durch Dependencies dynamisch ins lokale Repository geladen (\$HOME/.m2/repository)\\
		Bei einem Buildprozess durchläuft ein Projekt einen Lifecycle mit folgenden Phasen:
		\begin{description}
			\item[validate] validiert Projektdefinition
			\item[compile] Kompiliert die Quellen
			\item[test] Ausführen der Unit-Tests
			\item[package] Packen der Distribution
			\item[verify] Ausführen der Integrations-Tests
			\item[install] Deployment im lokalen Repository
			\item[deploy] Deployment im zentralen Repository
		\end{description}
		
	\section{Buildserver und CI}
\end{document}