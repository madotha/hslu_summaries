% Dokumentklassen:
% article, report, beamer, book, letter etc.
% https://en.wikibooks.org/wiki/LaTeX/Document_Structure
\documentclass[a4paper]{article}

% Seitenränder Abstand setzen
\usepackage[margin=80pt]{geometry}

% Deutsches Sprachpaket
\usepackage[ngerman]{babel}
% UTF8 Input Encoding
\usepackage[utf8]{inputenc}

% Schriftbild ändern
% https://en.wikibooks.org/wiki/LaTeX/Fonts
\usepackage[scaled]{helvet}
% (Sans) Serifen oder anderes
% \rmdefault: Serifen
% \sfdefault: Sans-Serifen
% \ttdefault: Typewriter
%\renewcommand{\familydefault}{\sfdefault}
% Fontencoding (für ä, ö, ü etc.)
\usepackage[T1]{fontenc}

% Gänsefüsschen richtig kompilieren
\usepackage [autostyle]{csquotes}
\MakeOuterQuote{"}

% Hyperlinks farblos
\usepackage[hidelinks]{hyperref}
\hypersetup{colorlinks=false}

% Package für Aufzählungen
\usepackage{enumitem}
% kein Abstand zwischen Aufzählungen
% Sollen doch Abstände vorhanden sein: nach Aufzählung {itemsep=1em}
\setlist{nosep}

% Grafik-Packages, für Figures, Subfigures und PDF als Import
\usepackage{graphicx}
\usepackage{subcaption}
\usepackage{pdfpages}

% Package und Einstellungen für Java-Code-Darstellung
% Werden erstellt mit \begin{lstlisting}
\usepackage{listings}
\usepackage{color}
\definecolor{dkgreen}{rgb}{0,0.6,0}
\definecolor{gray}{rgb}{0.5,0.5,0.5}
\definecolor{mauve}{rgb}{0.58,0,0.82}
\lstset{frame=tb,
	language=Java,
	aboveskip=3mm,
	belowskip=3mm,
	showstringspaces=false,
	columns=flexible,
	basicstyle={\small\ttfamily},
	numbers=none,
	numberstyle=\tiny\color{gray},
	keywordstyle=\color{blue},
	commentstyle=\color{dkgreen},
	stringstyle=\color{mauve},
	breaklines=true,
	breakatwhitespace=true,
	tabsize=3
}

\title{\textbf{Zusammenfassung MOBLAB} \\
		Mobile Programming Lab}
\date{\today}
\author{Maurin D. Thalmann}

\begin{document}
	
	\pagenumbering{gobble}
	\maketitle
	
	\newpage
	\pagenumbering{arabic}
	\tableofcontents
	
	\newpage
	
	\section{Tech-Intro}
	
		\subsection{Mobile Craftmanship Mindset}
		
		$\rightarrow$ \textbf{1:1 Portierung von Desktop zu Mobile reicht nicht aus!}
	
		\begin{itemize}
			\item Andere Benutzereingaben möglich auf Mobile: Touch, Pinch, Drag etc.
			\item Integrierte Sensoren: GPS, Kamera, Gyro, NFC, Bluetooth etc.
			\item Neue Einsatzmöglichkeiten: kontaktlose Interaktion, location-based, augmented etc.
		\end{itemize}
	
		\vspace{1em}
		
		\begin{itemize}
			\item Mindset der Entwickler \& Designer an neue Möglichkeiten anpassen
			\item Anforderungen \& Wünsche der Nutzer und des Markts prüfen (User-Interaction, Plattformstandards)
			\item Gute Entwickler kennen Plattformen, Betriebssysteme \& Bibliotheken
		\end{itemize}
	
		\subsection{Entwicklung mobiler Apps}
		
		\begin{figure}[htb!]
			\centering
			\includegraphics[width=0.7\textwidth]{img/techintro/spannungsfeld_apps.png}
			\caption{Das Spannungsfeld mobiler Entwicklung}
			\label{fig:techintro_spannungsfeld_apps}
		\end{figure}
	
		\begin{description}
			\item[Web-App] JS, HTML, CSS mit Responsive Design, im Browser ausgeführt
			\item[Hybrid] Web-App in nativem Wrapper verpackt, mit Connector-Plugins, kann als native App installiert werden (Cordova, Flutter etc.)
			\item[Cross-Compiled] In Sprache X geschriebene App, wird nach Java/Object-C oder binäres Format kompiliert und somit "native" (Xamarin, Ruby etc.)
			\item[JIT-compiled / VM] Javascript-App, läuft auf JS-Engine des Zielsystems und wird dort "just in time" kompiliert.
			Natives GUI und Konnektoren, um mit nativer Platform zu interagieren (NativeScript)
			\item[Native App] Spezifisch pro OS programmiert, nutzt volles Featureset der Plattform, auch neuste Features
		\end{description}
	
		\begin{figure}[htb!]
			\centering
			\includegraphics[width=.55\textwidth]{img/techintro/entwicklungsansätze.png}
			\caption{Übersicht mobiler Entwicklungsansätze}
			\label{fig:techintro_entwicklungsansätze}
		\end{figure}
	
		\newpage
	
		Welches ist jedoch der beste Ansatz für eine App?
		Immer abhängig von Anforderungen und Möglichkeiten.
		Mögliche Szenarien:
		
		\begin{itemize}
			\item Info-App ohne viel Interaktion $\rightarrow$ \textbf{Web-App}
			\item 100\% natives Look n Feel $\rightarrow$ \textbf{Nativ}
			\item Viele Plattformen nativ unterstützen, preiswert, .NET oder Angular Knowhow vorhande $\rightarrow$ \textbf{Cross-/JIT-Compile (Xamarin, NativeScript)}
			\item Gemeinsame Codebase, Crossplattform, eigene Widgets, Hot-Reload, kleine App / Prototyp $\rightarrow$ \textbf{Flutter}
			\item HW-Features benötigt auf verschiedenen Plattformen (NFC, Kamera, BT, Storage...) $\rightarrow$ \textbf{Hybrid / Nativ}
		\end{itemize}
	
		\subsection{Android Grundlagen}
		
		\begin{itemize}
			\item Android-Applikationen bestehen aus vier Komponenten:
			\begin{description}
				\item[Activity] UI-Komponente, entspricht typischerweise einem Bildschirm
				\item[Service] Komponente ohne UI, Dienst läuft typischerweise im Hintergrund
				\item[Broadcast Receiver] "Event-Handler", reagiert auf Broadcastsnachrichten (Intents)
				\item[Content Provider] Komponente, ermöglicht Datenaustausch zwischen versch. Applikationen
			\end{description}
			\item \textbf{Android Runtime (ART)} verwaltet die einzelnen Komponenten einer Applikation
			\begin{itemize}
				\item Mit Intent-Mechanismus kann eine Komponente eine andere Komponente aufrufen
				\item Komponenten müssen beim System registriert sein (teilweise Rechte = Privileges)
				\item System verwaltet Lebenszyklus von Komponenten (Gestartet, Pausiert, Aktiv, Gestoppt etc.)
			\end{itemize}
			\item Android empfiehlt für Hintergrundaufgaben nicht mehr Services, sondern \texttt{android.app.job.JobScheduler}\\
			Neu ist JobScheduler auch in WorkManager von Android Jetpack integriert
		\end{itemize}
	
			\subsubsection{Android Manifest}
			
			\begin{itemize}
				\item Beschreibt statische Eigenschaften einer Applikation
				\begin{itemize}
					\item Basis Java-Package-Name
					\item Benötigte Rechte (Internet, Kontakte etc.)
					\item Deklaration von Komponenten
					\begin{itemize}
						\item Activities, Services, Content Providers, Broadcast Receivers
						\item Name (+ Basis-Package = Java-Klasse)
						\item Anforderungen für Aufruf (Intent-Filter) für \textbf{A, S, BR}
						\item Format der gelieferten Daten für \textbf{CP}
					\end{itemize}
				\end{itemize}
				\item Diese Infos werden bei der App-Installation im System registriert
				\item Zusatzinfos (Version, ID etc.) befinden sich im Build-Skript (da diese build-abhängig sein können)
			\end{itemize}
		
			\subsubsection{Android Projekt-Struktur}
			
			\begin{itemize}
				\item Manifest
				\item Java-Code: Activities (App-Logik, Tests usw.)
				\item Ressourcen (\texttt{res})
				\begin{itemize}
					\item Bilder (\texttt{drawable})
					\item Layouts (\texttt{layout})
					\item Menus (\texttt{menu})
					\item Werte (\texttt{value})
				\end{itemize}
				\item Gradle Skripts (Angaben zum Build)
			\end{itemize}
		
		\newpage
		
		\subsection{Android Jetpack \& App-Architektur}
		
		Zwei massive Neuerungen in letzter Zeit:\\
		Seit 2019: Kotlin wird primäre Android-Sprache\\
		Seit 2018: Android Jetpack wird ins Leben gerufen
		
			\subsubsection{Android Jetpack}
			
			\begin{itemize}
				\item Sammlung von SW-Komponenten, die bei der Entwicklung von state-of-the-art Android-Applikationen unterstützen soll
				\item Jetpack-Komponenten im \texttt{androidx.}-Namespace, wurden teils aus Standard-API hierhin verschoben
				\item Alle Komponenten sind rückwärtskompatibel, können unabhängig von Android-Release-Zyklus aktualisiert und verwendet werden
				\item Jetpack wird von Google entwickelt und dokumentiert
			\end{itemize}
			\vspace{1em}
			Jetpack wird unterteilt in 4 Bereiche:
			\vspace{1em}
			\begin{itemize}
				\item Architecture
				\begin{itemize}
					\item Data Binding, Lifecycles, LiveData etc.
				\end{itemize}
				\item UI
				\begin{itemize}
					\item Animation/Transitions, Auto, TV, Wear, Emoji, Fragment etc.
				\end{itemize}
				\item Behavior
				\begin{itemize}
					\item Download Manager, Media \& Playback, Permissions, Notifications etc.
				\end{itemize}
				\item Foundation
				\begin{itemize}
					\item AppCompat, Android KTX, Multidex, Test
				\end{itemize}
			\end{itemize}
			
		\begin{figure}[htb!]
			\centering
			\includegraphics[width=0.5\textwidth]{img/techintro/apparch.png}
			\caption{Empfohlene App-Architektur}
			\label{fig:techintro_apparch}
		\end{figure}
		
		\newpage
		
			\subsubsection{Android Architecture Components (AAC)}
			
			\begin{itemize}
				\item Android Architecture Components enthalten eine Reihe von Lifecycle-bewussten Komponenten
				\item Komponenten helfen bei der Lösung von Problemen mit Konfigurationswechsel, Persistenz, Memory-Leaks und asynchronem Datenupdate auf dem UI
				\item AAC definieren seit 2017 eine standardisierte Vorgehensweise und stellen die offiziell empfohlene Lösung von Google dar
			\end{itemize}
		
			\begin{figure}[htb!]
				\centering
				\includegraphics[width=0.7\textwidth]{img/techintro/overview_aac.png}
				\caption{Übersicht der Android Architecture Components}
				\label{fig:techintro_overview_aac}
			\end{figure}
		
				\paragraph{Data Binding}
				
				\begin{itemize}
					\item Separiert das UI von den Daten
					\item Synchronisiert UI mit Daten (1-way oder 2-way Binding)
					\item Verwendet "Binding Expressions" mit @{...} Syntax im Layout-File, um View-Attribute zu initialisieren
				\end{itemize}
			
				\paragraph{Lifecycle}
				
				\begin{itemize}
					\item Kann Code aus den Lifecycle-Hooks von Activities entfernen und direkt auf der beobachtenden Komponente implementieren.
					\item \texttt{Lifecycle} ist ein Objekt, welches den Lebenszyklus einer Komponente abbildet (Activity, Fragment etc.)
					\item Andere Komponenten können das Lifecycle-Objekt beobachten und auf Lifecycle-Events reagieren
				\end{itemize}
			
				\paragraph{LiveData}
				
				\begin{itemize}
					\item Updates können im Hintergrund erfolgen, werden aber nur ausgeführt, wenn der Observer der LiveData in einem aktiven Zustand ist (\texttt{STARTED, RESUMED})
					\item \texttt{LiveData} ist eine lifecycle-aware Observable-Klasse
					\item Kann als Source für Data Binding verwendet werden, um Aktualisierungen des UI während der Laufzeit zu forcieren
				\end{itemize}
			
				\paragraph{Navigation}
				
				\begin{itemize}
					\item Android gibt neu Navigationsprinzipien vor, hierbei hilft die \texttt{Navigation}-Komponente bei der Implimentierung dieser Prinzipien
					\item \texttt{Navigation} basiert Navigation-Graph (Resources) mit Destinations (Knoten) und Actions (Kanten)
						\textit{(Navigation-Graph wird von Hand oder mit Navigation-Editor in Android Studio erstellt)}
					\item Navigation benötigt \texttt{NavHostFragment} im Layout (um die Zielfragmente einzublenden);
					Aus dem Code heraus wird mit einer \texttt{NavController}-Instanz navigiert
				\end{itemize}
			
				\paragraph{Paging}
				
				\begin{itemize}
					\item Es müssen nicht alle Daten auf einmal geladen werden: schneller und weniger Load
					\item Unterstützt asynchrones Laden von Daten
					\item \texttt{PagedList} und \texttt{PagedListAdapter}, um bei Bedarf weitere Daten in einer \texttt{RecyclerView} zu laden
				\end{itemize}
			
				\paragraph{Room}
				
				\begin{itemize}
					\item Bessere Abstraktion, Speichern/Laden von Modellobjekten, kein handgestricktes SQL-Mapping
					\item Room ist ein ORM (Object-Relational Mapper) für SQLite
					\item Arbeit mit Entities (Modell-Objekten) und DAO-Pattern anstatt SQL
				\end{itemize}
			
				\paragraph{ViewModel}
				
				\begin{itemize}
					\item Weniger Aufwand für Behandlung von Konfigurationsänderungen (keine Serialisierung nötig)
					\item Kapselt UI-Daten so, dass sie bei Konfigurationsänderung einer Activity in-memory erhalten bleiben
					\item Aber: Für den Fall eines App-Kills durch das OS müssen Daten immer noch persistiert werden!
				\end{itemize}
			
				\paragraph{WorkManager}
				
				\begin{itemize}
					\item Kein Kopfzerbrechen über Regeln für Hintergrundtasks, Synchronisation von UI via LiveData
					\item \texttt{WorkManager} führt asynchrone \texttt{WorkRequests} sofort oder zu geeignetem Zeitpunkt aus
					\item Respektiert Doze-Mode, versucht Ressourcen zu sparen und Load zu minimieren.
						Je nach Zustand von App/System werden Tasks unterschiedlich scheduled.
					\item Pro WorkRequest wird ein LiveData-Objekt erzeugt, Zustand und Daten sind darüber beobachtbar
				\end{itemize}\textbf{}
			
				\paragraph{App-Architektur: Tipps \& Empfehlungen}
				
				\begin{itemize}
					\item Standards / Patterns soweit möglich benutzen
					\item Aber: Kein Over-Engineering!
					\item Alle AAC können einzeln oder zusammen verwendet werden
					\item Herausfinden, was im Projekt am besten funktioniert bzw. am meisten Sinn macht
					\item Vorsicht bei Background-Tasks
					\begin{itemize}
						\item System zunehmend restriktiver, grosse Änderungen mit API 26
						\item "Background = Service" gilt nicht mehr
						\item WorkManager meist die bessere Alternative, Service für Logikexport 
					\end{itemize}
				\end{itemize}
				
	\newpage	
		
	\section{SA - Kotlin \& Android}
	
		\subsection{Sprachübersicht}
		
			\begin{itemize}
				\item Variablen zwingend mit \texttt{var} (veränderbar) oder \texttt{val} (unveränderbar) kennzeichnen
				\item Primitive Datentypen gibt es nicht, dafür Klassen für diese (Int, Double etc.)
				\item Kontrollstruktur-Ausdrücke haben immer einen Wert
				\item Keine checked Exceptions
				\item Strichpunkt am Ende einer Zeile ist optional
				\item Typinferenzen, weniger explizite Typangaben als bei Java
				\item Nullable-Typen können gegen NullPointerExceptions vorbeugen
				\item Mit Extensions Klassen ohne Vererbung erweitern
				\item Data Classes für kompakte Klassendeklaration inkl. automatisch implementierten equal / toString / hashCode Methoden
			\end{itemize}
	
		\subsection{Spracheigenschaften (eine Auswahl)}
		
			\paragraph{Variablen-Deklaration: var, val \& Typinferenz}
			\begin{itemize}
				\item \texttt{var} (veränderbar) und \texttt{val} (unveränderbar)
				\item Typangaben können grundsätzlich weggelassen werden \\
					(Compiler erkennt Datentyp automatisch und weist diesen korrekt zu)
				\item Java kennt für lokale Variablen seit 2018 ähnliche var-Syntax mit Typinferenz für lokale Variablen
				\item Bei Kotlin: Typangabe nach dem Namen\\
					Bei Java: Typangabe vor dem Namen
			\end{itemize}
		
			\paragraph{Nullable-Typen, Safe-Calls \& Elvis-Operator}
			\begin{itemize}
				\item In Kotlin können jegliche Typen nie den Wert null annehmen\\ 
					(ausser sie werden explizit als nullable deklariert)
				\item Variable, die Wert null zulassen soll, so deklarieren mit \texttt{[type]?}
				\item Nullable-Variablen können nicht direkt abgerufen werden (könnte ja null sein)
				\begin{itemize}
					\item Variable entweder erst auf null überprüfen
					\item Safe-Call Operator \texttt{?.} nutzen (druckt Wert der Variable oder null)
					\item Not-Null-Assertion-Operator \texttt{!!} (wirf NullPointerException, falls Wert null ist)
				\end{itemize}
				\item Spezialfall: falls nicht-null, nimm den Wert und sonst X
				\begin{itemize}
					\item Elvis-Operator \texttt{?:} (folgend auf diesen kann ein Wert angegeben werden)
					\item Zusammen mit Safe-Call-Operator nützlich, da so ein Alternativwert angegeben werden kann, falls eine Variable doch null sein sollte
				\end{itemize}
			\end{itemize}
		
			\paragraph{Funktionen: Benannte Parameter \& Default-Werte}
			\begin{itemize}
				\item Schlüsselwort \texttt{fun} zur Deklaration von Funktionen
				\item Benannte Parameter und Default-Werte für Parameter
				\begin{itemize}
					\item Benannte Parameter für bessere Lesbarkeit
					\item Default-Werte für übersichtliche Aufrufe
					\item Bei Aufruf einer Funktion mit bennanten Parametern können diese beliebig angeordnet werden, 
					jedoch darf beim Aufruf nicht zwischen unbenannt und benannt gemischt werden, da sonst die Position der Parameter nicht mehr stimmen könnte
				\end{itemize}
				\item Default-Werte helfen bei der Code-Einsparung (bei Java brucht es bei x Parametern x Methoden)
				\item Weitere spannende Möglichkeiten von Funktionen:\\
					\texttt{tailrec} für endrekursive Funktionen, Funktionen mit nur einem Ausdruck, Funktionen mit expliziten Rückgabetypen, Modifier \texttt{infix} für infix-Aufrufe (infix-Notation und ohne Klammern)
			\end{itemize}
		
			\paragraph{Extension: Erweiterung ohne Vererbung}
			\begin{itemize}
				\item Extensions, um bestehende oder auch fremde Klassen um Funktionen zu erweitern\\
					z.Bsp. kann Klasse \texttt{Int} um Funktion \texttt{myPrettyPrint()} erweitert werden
			\end{itemize}
	
		\newpage	
	
		\subsection{Kotlin \& Android}
		
		\begin{itemize}
			\item Kotlin offizielle Androidsprache, da kompakt, ausdrucksstark, Typ- und Null-sicher
			\item Parallele Verwendung von Java und Kotlin möglich für Android-Entwicklung
			\item Aus Kotlin-Klasse kann auf Java-Klasse zugegriffen werden (und umgekehrt)
			\item Sicherer als Java dank non-nullable Typen und kompakteren Konstrukten \\
				40\% weniger Zeilen Code als bei Java
			\item Android findViewById() entfällt bei Kotlin, da direkt Extension-Properties für alle id's in Ressourcen-Dateien angelegt werden, so kann direkt auf jede id zugegriffen werden
		\end{itemize}
		
	\section{SA - Data Binding \& ViewModel}
	
	
	
	\section{SA - Fastlane}
	
	
	
	\section{SA - Unreal Engine}
	
	
	
	\section{SA - Xamarin.Forms}
	
	
	
	\section{SA - PWA: Progressive Web Apps}
	
\end{document}
